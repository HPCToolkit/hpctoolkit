%
% $Id$
%
\documentclass[12pt]{article}

\usepackage{fancyvrb}
\usepackage{fullpage}
\usepackage{moreverb}
\usepackage{hyperref}
\usepackage{verbatim}

\setlength{\topmargin}{0in}
\setlength{\oddsidemargin}{0in}
\setlength{\evensidemargin}{0in}
\setlength{\textwidth}{6.5in}
% \setlength{\parskip}{4pt}
% \setlength{\parindent}{0pt}

\newcommand{\HPCToolkit}{\textsc{HPCToolkit}}
\newcommand{\hpcrun}{\texttt{hpcrun}}
\newcommand{\hpclink}{\texttt{hpclink}}
\newcommand{\hpcstruct}{\texttt{hpcstruct}}
\newcommand{\hpcprof}{\texttt{hpcprof}}
\newcommand{\hpcviewer}{\texttt{hpcviewer}}

\newcommand{\question}[1]{\vspace{4pt}\par\noindent{\bf Q: #1}}
\newcommand{\answer}{\par\vspace{2pt}\noindent{\bf A:}}

\begin{document}

\title{Using {\sc HPCToolkit} with Statically Linked Programs}
\author{The \HPCToolkit\ Team}
\maketitle

\begin{abstract}
\HPCToolkit{} is an integrated suite of tools that supports
measurement, analysis, attribution and presentation of application
performance for sequential and parallel programs.  This document
describes how to link and run statically linked applications so that you can use
 \HPCToolkit\'s monitoring capabilities. 
 \end{abstract}

\section{Introduction}
On modern Linux systems, dynamically linked executables are the default. With dynamically linked executables, \HPCToolkit's \hpcrun{} script uses  library preloading
 to inject \HPCToolkit's monitoring code into an application's address space. 
However, in some cases, one wants or needs to build statically linked executables. 
\begin{itemize}
\item One might want to build a statically linked executable because they are generally faster if the executable spends a significant amount of time calling functions in libraries. 
\item On scalable parallel systems such as a Blue Gene/P or a Cray XT, at present the compute node kernels don't support dynamically linked executables; for these systems, one needs to build statically linked executables.
\end{itemize}

For statically linked executables, preloading \HPCToolkit's monitoring code into an application's address space at program launch is not an option. Instead, monitoring code must be added at link time; \HPCToolkit{}'s \hpclink{} script is used for this purpose.

\section{Linking with \hpclink }

Adding \HPCToolkit's monitoring
code into a statically-linked application is easy.  This does not require any
source-code modifications, but it does involve a small change to your
build procedure.  You continue to compile
all of your object ({\tt .o}) files exactly as before, but you will need to 
modify your final link step
to use \hpclink\ to add  \HPCToolkit's monitoring code to your executable.

In your build scripts, locate the last step in the build, namely,  the
command that produces the final statically linked binary.  Edit that command line to add the \hpclink\ command at the front.

For example, suppose that the name of your executable is {\tt myprog} and the last step in
your {\tt Makefile} links various object files and libraries as
follows into a statically linked executable:
%
\begin{quote}
\verb|mpicc -o myprog -static file.o ... -l<lib> ...|
\end{quote}
%
To build a version of your executable with \HPCToolkit's monitoring code linked in, you would use the
following command line:
%
\begin{quote}
\verb|hpclink mpicc -o myprog -static file.o ... -l<lib> ...|
\end{quote}

In practice, you may want to edit your {\tt Makefile} to always build
two versions of your program, perhaps naming them {\tt myprog} and
{\tt myprog.hpc}.

\section{Running the Statically-Linked Binary}

For dynamically linked executables, the \hpcrun\ script sets environment variables to
pass information to the \HPCToolkit\ monitoring library. On standard Linux systems,  statically-linked executables can still be launched with \hpcrun. 

On Cray XT and Blue Gene/P systems,  the  \hpcrun\   script is not applicable because of differences in application launch procedures.
On these systems, you will need to use the {\tt CSPROF\_OPT\_EVENT} environment variable to pass \HPCToolkit's monitoring code, which was linked into your executable using \hpclink, a  list of events to be monitored.
Typically, you would set {\tt CSPROF\_OPT\_EVENT}  in your launch script.

The {\tt CSPROF\_OPT\_EVENT} environment variable
should be set to a space-separated list of {\tt EVENT@COUNT} pairs.
For example, in a PBS script for a Cray XT system, you might write the following in Bourne shell or bash syntax:

\begin{quote}
\begin{verbatim}
#!/bin/sh
#PBS -l size=64
#PBS -l walltime=01:00:00
cd $PBS_O_WORKDIR
export CSPROF_OPT_EVENT="PAPI_TOT_CYC@4000000 PAPI_L2_TCM@400000"
aprun -n 64 ./myprog arg ...
\end{verbatim}
\end{quote}
% $ Artificially end math mode.

Using the Cobalt job launcher on Argonne National Laboratory's BG/P system, you would use  the {\tt --env} option to pass environment
variables.  For example, you might submit a job with:

\begin{quote}
\begin{verbatim}
qsub -t 60 -n 64 --env CSPROF_OPT_EVENT="PAPI_TOT_CYC@4000000" \
    /path/to/myprog arg ...
\end{verbatim}
\end{quote}

\section{Troubleshooting}

With some compilers you need to disable interprocedural
optimization to use \hpclink.  To instrument your statically linked executable at link time, \hpclink\ inserts monitoring hooks into  your program
with the {\tt ld} option {\tt --wrap} (see the ld(1) man page).  For some compilers, {\em e.g.}, IBM's XL compilers and Pathscale's compilers,
interprocedural optimization interferes with the {\tt --wrap} option
and prevents \hpclink\ from working properly. If this is the case, \hpclink{} will emit error messages and fail. If you want to use \hpclink\ with such compilers, sadly, you must turn off interprocedural optimization. 

Note that interprocedural optimization may not be explicitly enabled during your compiles; it might be implicitly enabled when using a compiler optimization option such as {\tt -fast}. In cases such as this, you can often specify {\tt -fast} along with an option such as {\tt -no-ipa}; this option combination will provide the benefit of all of {\tt -fast}'s optimizations {\em except} interprocedural optimization.

\end{document}
