%\documentclass[12pt]{article}
\documentclass{article}
\usepackage{hyperref}
\usepackage{verbatim,moreverb,fancyvrb}
\setlength{\topmargin}{0in}
\setlength{\oddsidemargin}{0in}
\setlength{\evensidemargin}{0in}
\setlength{\textwidth}{6.5in}
\setlength{\parindent}{0in}

% ----------------------------------------------------------
% tools
% ----------------------------------------------------------

\newcommand{\hpctoolkit}{\textsc{HPCToolkit}}
\newcommand{\hpcrun}{\texttt{hpcrun}}
\newcommand{\hpcstruct}{\texttt{hpcstruct}}
\newcommand{\hpcprof}{\texttt{hpcprof}}
\newcommand{\hpcviewer}{\texttt{hpcviewer}}

% ----------------------------------------------------------
% convenient abbreviations
% ----------------------------------------------------------

\newcommand{\eg}{{\em e.g.}}


\begin{document}
\title{Understanding and Troubleshooting \\
         \hpctoolkit\ Problems}
\author{The \hpctoolkit\ Team}
\date{14 April 2009}
\maketitle


\textbf{When I run \hpcviewer, I only see results per function call,
not per line of code.}
\begin{quote}
Your program probably lacks debugging information or else you have
stripped the binary.  The debugging information includes a line map
which is used by profilers and debuggers to match up machine addresses
with source code lines.  \hpctoolkit\ can profile binaries without
debugging information, but in that case, it can only display the
results per function call, not per source line.

The solution is to always compile your programs with debugging
information.  The option for this varies by compiler.  We suggest the
following options: for the GNU compilers (gcc, gfortran) use {\tt -g},
for the Intel compilers (icc, ifort) use ``{\tt -g -debug
inline\_debug\_info}'', for the Pathscale compilers (pathcc, pathf95)
use {\tt -g1}, and for the PGI compilers (pgcc, pgf95) use {\tt
-gopt}.  Also, be careful not to strip the binary as that would remove
the debugging information.

{\it Note:} Adding debugging information does not make a program run
slower, and stripping the binary does not make it run faster.  In
general, debugging information is compatible with compiler
optimization.  Just understand that at high optimization levels, the
compiler may make significant program transformations which do not
cleanly map to line numbers in the original source code.  Also, be
sure to explicitly add any optimization options to the compile line,
preferably after the debugging options, for example, ``{\tt -g -O2}''.
Many compilers will lower the optimization level if the compile line
includes debugging options but no optimization option.
\end{quote}


\textbf{\hpcviewer\ runs glacially slow, what is the workaround?}
\begin{quote}
There are three likely reasons why \hpcviewer{} might run terribly
slow.  You may be running \hpcviewer{} on a remote system with low
bandwidth, high latency or an otherwise unsatisfactory network
connection to your desktop. If any of these conditions are true,
\hpcviewer{}'s otherwise snappy GUI can become sluggish if not
downright unresponsive.  The solution is to install \hpcviewer{} on
your local system, copy the database onto your local system, and run
\hpcviewer{} locally. We almost always run \hpcviewer{} on our local
workstations or laptops for this reason.

Another reason could be that the {\tt experiment.xml} file in your
database is very large.  If this is tens of megabytes or more, the
total database size might be the problem.

This problem can also occur if the database contains too many columns
of metrics.  This can happen if you use \hpcprof{} to build a database
for several threads with several metrics each, resulting in too many
metrics total. You can check the number of columns in your database by
running
\begin{quote}
\verb,grep -e "<Metric" experiment.xml | wc -l,
\end{quote}
If that command yields a number greater than 30 or so, \hpcviewer{} is
likely slow because you are working with too many columns of metrics.
In this case, run \hpcprof\ to build a database with fewer threads.
\end{quote}


\textbf{Why isn't my source code found by \hpcviewer ?}
\begin{comment}
(circular links in recursive search may lead to  inf loop)
\end{comment}
\begin{quote}
When using \hpcprof\ to create performance databases, one must specify the path 
to the source directories using the \verb|-I dir| flag. If there are multiple
source directories, then multiple \verb|-I| flags are required:
\begin{quote}
 \verb|-I dir1 -I dir2 ... -I dirN|
\end{quote}
The specified list of directory paths, \verb|dir1| through
\verb|dirN|, can be either relative or absolute.

In addition, \hpcprof\ has a recursive directory search feature that
can greatly reduce the number of \verb|-I| flag instances. To conduct
a source-file search of a directory \emph{and all of it's descendants},
simply append an escaped '*' after the last slash, e.g.,
\verb|/mypath/\*| ( or \verb|/mypath/'*'| ). This greatly simplifies
the \hpcprof\ command line for a typical project. For example, suppose
a project has all of it's source in a top-level directory called
\verb|src|. Inside \verb|src| are several libraries ( like
\verb|src/lib1|,\verb|src/lib2|, \verb|src/lib-math|, etc). Then, all
of the source is visible to \hpcprof\ by using the single flag
\begin{quote}
\verb|-I src/'*'|
\end{quote}

\textbf{NOTE:} The \verb|'*'| can be used \emph{only} at the end of
a directory path.

For any functions whose source code is not found in the specified directories
(e.g., system libraries), \hpcviewer\ will generate a synopsis that shows
the presence of the function and its line extents (if known), but no
source code.
\end{quote}

\vbox{%
\textbf{I tried the approach above, but I am still missing source files.}
\begin{quote}
This is a common problem. The cause for this difficulty is that the
pathnames associated with performance metrics are extracted from an
application binary. The path names encoded in the binary can be

\begin{itemize}
   \item relative to the directory where the source was compiled, not
         where you are now as you measure and analyze the code, or

   \item absolute paths that have encodings for the mount point the
         directory that existed when the file was compiled, but might be
         different now (e.g., they were compiled on a different node of a
         cluster that had a different mount point for the shared file
         system).
\end{itemize}
\end{quote}
} 
\begin{quote}
Diagnosing and fixing this problem requires knowing exactly what path
names are referenced in the binary and/or perhaps the performance
data. Fortunately, this is information is supplied by \hpcprof .
If a source file is successfully located, then a 
\begin{quote}
\verb|msg:   cp:...|
\end{quote}
line appears in the output of \hpcprof . Unlocated files are deemed 'lost'
and there is an output line of the form 
\begin{quote}
\verb|WARNING:  lost:|
\end{quote}
in the output.

For example, suppose we have an application \verb|app1| whose main source 
is in in a directory \verb|/projs/Apps/app1-src|. The \verb|app1|
application is built inside the \verb|app1-src| subdirectory, and it uses
source files from a subdirectory \verb|app1-src/special| as well as some
source common to all applications, located in
\verb|/projs/Apps/common|. When \verb|app1| is built, the
\verb|common| source is accessed by relative path \verb|../common|.
The \verb|app1| executable is installed on our path.

Now, we switch to our home directory \verb|/h/user/T1| to collect
some profile data for \verb|app1|.
When we run \hpcprof\ (without the \verb|-I| flag) as follows:
\begin{quote}
  \verb|hpcprof -S app1.hpcstruct hpctoolkit-measurements/|
\end{quote}
This results in the output
\begin{quote}
\begin{Verbatim}[fontsize=\small]
msg: Line map : /opt/apps/intel/compilers/10.1/lib/libimf.so
msg: STRUCTURE: /usr/local/bin/app1
msg: Copying source files reached by PATH option to /h/user/T1/hpctoolkit-app1-database
WARNING: lost: app1.c
WARNING: lost: special/xfn1.c
WARNING: lost: ../common/mathx.c
WARNING: lost: ~unknown-file~
WARNING: lost: irc_msg_support.c
\end{Verbatim}
\end{quote}
The \verb|WARNING: lost:| obtains for \verb|~unknown-file~| and
\verb|irc_msg_support.c| because these are compiler system files --- source
is unavailable. The other lost files, however, can be found by using
the proper \verb|-I| flag:
\begin{quote}
\begin{verbatim}
hpcprof -I /projs/Apps/'*' -S app1.hpcstruct hpctoolkit-measurements/
\end{verbatim}
\end{quote}

The resulting output:
\begin{quote}
\begin{Verbatim}[fontsize=\small]
msg: Line map : /opt/apps/intel/compilers/10.1/lib/libimf.so
msg: STRUCTURE: /usr/local/bin/app1
msg: Copying source files reached by PATH option to /h/user/T1/hpctoolkit-app1-database
msg:   cp:/projs/Apps/app1-src/app1.c -> ./projs/Apps/app1-src/app1.c
msg:   cp:/projs/Apps/app1-src/special/xfn1.c -> ./projs/Apps/app1-src/special/xfn1.c
msg:   cp:/projs/Apps/common/mathx.c -> ./projs/Apps/common/mathx.c
WARNING: lost: ~unknown-file~
WARNING: lost: irc_msg_support.c
\end{Verbatim}
\end{quote}
\nobreak Much better!


\textbf{Best Practice:} First, carefully inspect the output of \hpcprof\ to
determine which files are lost. Next, determine the absolute path for
each distinct top-level source directory (or build directory, if it 
is separate from the source directory). Finally, for each of these (absolute) directory paths,
specify a \verb|-I| option with the recursive search option ( \verb|'*'| at the end
of the path).

\end{quote}

\begin{comment}
   possible question: What happens if I forget to run hpcstruct?  
No hpcstruct = 
   really crazy procedure & loop bounds
   no inlining detection!
\end{comment}

%\vbox

For procedures, often no machine instructions are associated with a
procedure's declarations. Thus, the first line in the procedure that
has an associated machine instruction is the first line of executable
code.

Inlined functions may occasionally lead to confusing data for a
procedure. Machine instructions mapped to source lines from the inline
function appear in the context of other functions. While \hpcprof 's
methods for handling inline functions are good, some codes can confuse
the system.

For loops, the process of identifying what source lines are in a loop
is similar to the procedure process: what source lines map to machine
instructions inside a loop defined by a backward branch to a loop
head. Sometimes compilers don't properly record the line number
mapping.

When the compiler line mapping information is wrong, there is little
you can do about it other than to ignore its imperfections, or
hand-edit the XML program structure file produced by \hpcstruct . This
technique is used only when truly desperate.
\end{quote}

\textbf{A particular scope in my code (\eg, a loop) contains one call to a 
function, yet \hpcviewer{} shows several. Is something wrong with \hpctoolkit{}?}
\begin{quote}
In a word: no. In the course of code optimization, compilers often replicate code blocks. 
For instance, as it generates code, a compiler may peel iterations from a loop or split the iteration 
space of a loop into two or more loops. 
In such cases, one call in the source code may be transformed into multiple 
distinct calls that reside at different code addresses in the executable. 

When analyzing applications at the binary level, it is difficult to 
determine whether two distinct calls to the same function that appear in the machine 
code were derived from the same call in the source code. 
Even if both calls map to the same source line, it may be wrong to coalesce them; 
the source code might contain multiple calls to the same function on the same line. 
By design, \hpctoolkit{} does not attempt to coalesce distinct calls to the same function
because it might be incorrect to do so; instead, it independently reports each call site 
that appears in the machine code. 
If the compiler duplicated calls as it replicated code during optimization,  
multiple call sites may be reported by \hpcviewer{} when only one appeared in the source code. 
\end{quote}

\end{document}
