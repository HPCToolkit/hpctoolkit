%% $Id$

%%%%%%%%%%%%%%%%%%%%%%%%%%%%%%%%%%%%%%%%%%%%%%%%%%%%%%%%%%%%%%%%%%%%%%%%%%%%%
%%%%%%%%%%%%%%%%%%%%%%%%%%%%%%%%%%%%%%%%%%%%%%%%%%%%%%%%%%%%%%%%%%%%%%%%%%%%%

\documentclass[english]{article}
\usepackage[latin1]{inputenc}
\usepackage{babel}
\usepackage{verbatim}

%% do we have the `hyperref package?
\IfFileExists{hyperref.sty}{
   \usepackage[bookmarksopen,bookmarksnumbered]{hyperref}
}{}

%% do we have the `fancyhdr' or `fancyheadings' package?
\IfFileExists{fancyhdr.sty}{
\usepackage[fancyhdr]{latex2man}
}{
\IfFileExists{fancyheadings.sty}{
\usepackage[fancy]{latex2man}
}{
\usepackage[nofancy]{latex2man}
\message{no fancyhdr or fancyheadings package present, discard it}
}}

%% do we have the `rcsinfo' package?
\IfFileExists{rcsinfo.sty}{
\usepackage[nofancy]{rcsinfo}
\rcsInfo $Id$
\setDate{\rcsInfoLongDate}
}{
\setDate{ 2009/02/06}
\message{package rcsinfo not present, discard it}
}

\setVersionWord{Version:}  %%% that's the default, no need to set it.
\setVersion{=PACKAGE_VERSION=}

%%%%%%%%%%%%%%%%%%%%%%%%%%%%%%%%%%%%%%%%%%%%%%%%%%%%%%%%%%%%%%%%%%%%%%%%%%%%%
%%%%%%%%%%%%%%%%%%%%%%%%%%%%%%%%%%%%%%%%%%%%%%%%%%%%%%%%%%%%%%%%%%%%%%%%%%%%%

\begin{document}

\begin{Name}{1}{hpctoolkit}{The HPCToolkit Performance Tools}{The HPCToolkit Performance Tools}{HPCToolkit}

\textbf{HPCToolkit} is a collection of performance analysis tools for node-based performance analysis.
It has been designed around the following principles:
\begin{itemize}

\item \textbf{Language independence.}
\item \textbf{Avoid code instrumentation.}
\item \textbf{Avoid blind spots.}
\item \textbf{Context is essential for understanding layered and object-oriented software.}
\item \textbf{Any one performance measure produces a myopic view.}
\item \textbf{Derived performance metrics are essential for effective analysis.}
\item \textbf{Performance analysis should be top down.}
\item \textbf{Hierarchical aggregation is important in the face of approximate attribution.} 
\item \textbf{With instruction-level parallelism, aggregate properties are vital.}
\item \textbf{Measurement and analysis must be scalable.}

\end{itemize}

\textbf{HPCToolkit}'s website contains papers that explain these design principles in more detail.

\end{Name}

%%%%%%%%%%%%%%%%%%%%%%%%%%%%%%%%%%%%%%%%%%%%%%%%%%%%%%%%%%%%%%%%%%
\section{Description}

\subsection{Top-Down Analysis With Calling Context and Program Structure}

A typical node-based performance analysis session consists of:
\begin{enumerate}
\item \textbf{Measurement.}  
Collect low-overhead, high-accuracy profiles using statistical sampling.
\HTMLhref{hpcrun.html}{\Cmd{hpcrun}{1}} collects call path profiles while \HTMLhref{hpcrun-flat.html}{\Cmd{hpcrun-flat}{1}} collects `flat' profiles (IP histograms, where IP is instruction pointer).

\item \textbf{Recovering static source code structure.} 
\HTMLhref{hpcstruct.html}{\Cmd{hpcstruct}{1}} recovers static program structure such as procedures and loop nests.
It accounts for procedure and loop transformations such as inlining and software pipelining.
Technically, this is an optional step, but critical information is lost without it.

\item \textbf{Correlating dynamic profiles with static source code structure.} 
\textbf{HPCToolkit}'s correlation tool combines dynamic profile information with \HTMLhref{hpcstruct.html}{\Cmd{hpcstruct}{1}}'s static program structure to correlate costs of the optimized object code to useful source code constructs such as loop nests.
The result of correlation is called an Experiment database.
Currently, \HTMLhref{hpcprof.html}{\Cmd{hpcprof}{1}} is used for call stack profiles and \HTMLhref{hpcprof-flat.html}{\Cmd{hpcprof-flat}{1}} is used for flat profiles.

\item \textbf{Top-down visualization}
\HTMLhref{hpcviewer.html}{\Cmd{hpcviewer}{1}} is a Rich Client Platform-based tool for presenting the resulting Experiment databases.
An important feature of the Experiment database is that it is relocatable, containing profile information and copies of application source files.
This means that the first three steps can be performed remotely on a cluster and then the database can be viewed locally on a laptop or workstation.
\end{enumerate}

\subsection{Textual Analysis}

\begin{enumerate}
\item \textbf{Measurement.}  
Collect low-overhead, high-accuracy `flat' profiles using statistical sampling (\HTMLhref{hpcrun-flat.html}{\Cmd{hpcrun-flat}{1}}).

\item \textbf{Recovering static source code structure.} 
\HTMLhref{hpcstruct.html}{\Cmd{hpcstruct}{1}} is used for the same purpose described above.

\item \textbf{Correlating dynamic profiles with procedures source lines.} 
\HTMLhref{hpcproftt.html}{\Cmd{hpcproftt}{1}} correlates `flat' profiles with source structure and produces textual output.

\end{enumerate}

%%%%%%%%%%%%%%%%%%%%%%%%%%%%%%%%%%%%%%%%%%%%%%%%%%%%%%%%%%%%%%%%%%
\section{Examples}

\subsection{Interactively visualize call path or flat metrics correlated with program structure}



\begin{enumerate}

\item \textbf{Compile}.  

Assume we have an application called \emph{zoo} whose source code is located in in \File{path-to-zoo}.
First we compile and link the application normally with full optimization and as much debugging information as possible.
Typically, this involves compiler options similar to \verb+-O3 -g+.

\item \textbf{Measure}.
Profile with \HTMLhref{hpcrun.html}{\Cmd{hpcrun}{1}}.  Assume we wish to use two different sets of events and that the results are placed in \File{profile-file-1} and \File{profile-file-2}.
\begin{verbatim}
  hpcrun <event-set-1> zoo
  hpcrun <event-set-2> zoo
\end{verbatim}

\item \textbf{Recover static source code structure}. 
Use \HTMLhref{hpcstruct.html}{\Cmd{hpcstruct}{1}} to recover program structure and write it to the file \File{zoo.hpcstruct}
\begin{verbatim}
  hpcstruct zoo
\end{verbatim}

\item \textbf{Correlate call path or flat metrics with static source code structure}.
Create an Experiment database using \HTMLhref{hpcprof-flat.html}{\Cmd{hpcprof-flat}{1}}.  Assume the Experiment database is located in \File{hpctoolkit-database}.
\begin{verbatim}
  hpcprof -I 'path-to-zoo' -S zoo.hpcstruct profile-file-1 profile-file-2
\end{verbatim}
or
\begin{verbatim}
  hpcprof-flat -I 'path-to-zoo/*' -S zoo.hpcstruct profile-file-1 profile-file-2
\end{verbatim}

\item \textbf{Visualize}.
Visualize the Experiment database using \HTMLhref{hpcviewer.html}{\Cmd{hpcviewer}{1}}:
\begin{verbatim}
  hpcviewer hpctoolkit-database
\end{verbatim}
Derived metrics may be computed on-the-fly with the viewer.

\end{enumerate}


\subsection{Visualize textual summaries of flat metrics}

Begin with steps 1 and 2 above.

\begin{itemize}

\item \textbf{Correlate metrics with static source code structure and generate textual summaries}.

To compute raw metrics for each native event and generate the default program and load module summaries:
\begin{verbatim}
  hpcproftt -I 'path-to-zoo/*' -S zoo.hpcstruct profile-file-1 profile-file-2
\end{verbatim}

To compute raw and summary metrics (but only show the latter) and generate summaries for all program structure elements:
\begin{verbatim}
  hpcproftt --src=sum --metric=sum-only -I 'path-to-zoo/*' -S zoo.hpcstruct profile-file-1 profile-file-2
\end{verbatim}

To compute raw and summary metrics and generate summaries for all program structure elements and generate annotated source files:
\begin{verbatim}
  hpcproftt --src=all --metric=sum -I 'path-to-zoo/*' -S zoo.hpcstruct profile-file-1 profile-file-2
\end{verbatim}


The first computes

\end{itemize}


\subsection{Interactively visualize derived flat metrics (computed in batch) correlated with program structure}

Begin with steps 1, 2 and 3 above.
These steps create a template \HTMLhref{hpcprof-flat.html}{\Cmd{hpcprof-flat}{1}} configuration file which can be modified and used to rerun \Prog{hpcprof-flat}.

\begin{itemize}
\item \textbf{Create derived metrics}.  Add COMPUTE metrics to the \HTMLhref{hpcprof-flat.html}{\Cmd{hpcprof-flat}{1}} configuration file (named \File{config.xml}) generated during the initial run.  The \HTMLhref{hpcprof-flat.html}{\Cmd{hpcprof-flat}{1}} man page contains some examples. 

\item \textbf{Correlate}. Rerun \HTMLhref{hpcprof-flat.html}{\Cmd{hpcprof-flat}{1}} to create a new Experiment database.  Assume that it is located in \File{hpctoolkit-database-1}
\begin{verbatim}
  hpcprof-flat --config config.xml
\end{verbatim}

\item \textbf{Visualize}.
Visualize the Experiment database using \HTMLhref{hpcviewer.html}{\Cmd{hpcviewer}{1}}:
\begin{verbatim}
  hpcviewer hpctoolkit-database-1/experiment.xml
\end{verbatim}

\end{itemize}


%%%%%%%%%%%%%%%%%%%%%%%%%%%%%%%%%%%%%%%%%%%%%%%%%%%%%%%%%%%%%%%%%%
%\section{Notes}

%%%%%%%%%%%%%%%%%%%%%%%%%%%%%%%%%%%%%%%%%%%%%%%%%%%%%%%%%%%%%%%%%%
\section{See Also}

\HTMLhref{hpcrun.html}{\Cmd{hpcrun}{1}}, \HTMLhref{hpcrun-flat.html}{\Cmd{hpcrun-flat}{1}} \\
\HTMLhref{hpcstruct.html}{\Cmd{hpcstruct}{1}} \\
\HTMLhref{hpcprof.html}{\Cmd{hpcprof}{1}}, \HTMLhref{hpcprof-flat.html}{\Cmd{hpcprof-flat}{1}}, \HTMLhref{hpcproftt.html}{\Cmd{hpcproftt}{1}} \\
%\Cmd{xprof}{1} 
\HTMLhref{hpcviewer.html}{\Cmd{hpcviewer}{1}}

%%%%%%%%%%%%%%%%%%%%%%%%%%%%%%%%%%%%%%%%%%%%%%%%%%%%%%%%%%%%%%%%%%
\section{Version}

Version: \Version\ of \Date.

%%%%%%%%%%%%%%%%%%%%%%%%%%%%%%%%%%%%%%%%%%%%%%%%%%%%%%%%%%%%%%%%%%
\section{License and Copyright}

\begin{description}
\item[Copyright] \copyright\ 2002-2008, Rice University.
\item[License] See \File{README.License}.
\end{description}

%%%%%%%%%%%%%%%%%%%%%%%%%%%%%%%%%%%%%%%%%%%%%%%%%%%%%%%%%%%%%%%%%%
\section{Authors}

Email: \Email{hpc@cs.rice.edu} \\
WWW: \URL{http://hipersoft.cs.rice.edu/hpctoolkit}.

\LatexManEnd

\end{document}

%% Local Variables:
%% eval: (add-hook 'write-file-hooks 'time-stamp)
%% time-stamp-start: "setDate{ "
%% time-stamp-format: "%:y/%02m/%02d"
%% time-stamp-end: "}\n"
%% time-stamp-line-limit: 50
%% End:

