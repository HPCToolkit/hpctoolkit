%% $Id$

%%%%%%%%%%%%%%%%%%%%%%%%%%%%%%%%%%%%%%%%%%%%%%%%%%%%%%%%%%%%%%%%%%%%%%%%%%%%%
%%%%%%%%%%%%%%%%%%%%%%%%%%%%%%%%%%%%%%%%%%%%%%%%%%%%%%%%%%%%%%%%%%%%%%%%%%%%%

\documentclass[english]{article}
\usepackage[latin1]{inputenc}
\usepackage{babel}
\usepackage{verbatim}

%% do we have the `hyperref package?
\IfFileExists{hyperref.sty}{
   \usepackage[bookmarksopen,bookmarksnumbered]{hyperref}
}{}

%% do we have the `fancyhdr' or `fancyheadings' package?
\IfFileExists{fancyhdr.sty}{
\usepackage[fancyhdr]{latex2man}
}{
\IfFileExists{fancyheadings.sty}{
\usepackage[fancy]{latex2man}
}{
\usepackage[nofancy]{latex2man}
\message{no fancyhdr or fancyheadings package present, discard it}
}}

%% do we have the `rcsinfo' package?
\IfFileExists{rcsinfo.sty}{
\usepackage[nofancy]{rcsinfo}
\rcsInfo $Id$
\setDate{\rcsInfoLongDate}
}{
\setDate{ 2008/05/30}
\message{package rcsinfo not present, discard it}
}

\setVersionWord{Version:}  %%% that's the default, no need to set it.
\setVersion{=PACKAGE_VERSION=}

%%%%%%%%%%%%%%%%%%%%%%%%%%%%%%%%%%%%%%%%%%%%%%%%%%%%%%%%%%%%%%%%%%%%%%%%%%%%%
%%%%%%%%%%%%%%%%%%%%%%%%%%%%%%%%%%%%%%%%%%%%%%%%%%%%%%%%%%%%%%%%%%%%%%%%%%%%%

\begin{document}

\begin{Name}{1}{hpctoolkit}{The HPCToolkit Performance Tools}{The HPCToolkit Performance Tools}{HPCToolkit}

\textbf{HPCToolkit} is a collection of performance analysis tools for node-based performance analysis.
It has been designed around the following principles:
\begin{itemize}

\item \textbf{Language independence.}
Modern scientific codes are multi-lingual; therefore \textbf{HPCToolkit} is not dependent on particular language.

\item \textbf{Avoid code instrumentation.}
Adding instrumentation manually requires making \emph{a priori} assumptions about bottlenecks; adding instrumentation automatically can dilate execution and preclude compiler optimizations.
Consequently, \textbf{HPCToolkit} uses a sampling based measurement approach.

\item \textbf{Avoid blind spots.}
Production applications frequently link against fully optimized and even partially stripped binaries such as math and communication libraries for which source code is not available.
Although such libraries pose significant measurement challenges, \textbf{HPCToolkit} employs techniques that enable it to collect performance measurements for such libraries.

\item \textbf{Context is essential for understanding layered and object-oriented software.}
In large, modular programs costs must be understood in their calling context.
Thus, \textbf{HPCToolkit} has a sophisticated call path profiler.

\item \textbf{Any one performance measure produces a myopic view.}
\textbf{HPCToolkit} supports collection, correlation and presentation of multiple metrics.

\item \textbf{Derived performance metrics are essential for effective analysis.}
Effective tuning a program with appropriate complexity bounds requires a measure of not where resources are consumed, but where they are consumed inefficiently.
For this reason, \textbf{HPCToolkit} supports derived metrics such as the differences between peak and actual performance 

\item \textbf{Performance analysis should be top down.}
To make analysis of large programs tractable, performance tools should organize performance data in a hierarchical fashion, prioritize what appear to be important problems, and support a top-down analysis methodology that helps users quickly locate bottlenecks.

\item \textbf{Hierarchical aggregation is important in the face of approximate attribution.} 
In modern multi-issue microprocessor cores with multiple functional units, out-of-order execution, and non-blocking caches, line level (or finer) information can be misleading.
For this reason, \textbf{HPCToolkit} focuses on aggregate information for loops or procedures, which can be very accurate.
Loop-level information is particularly useful.

\item \textbf{With instruction-level parallelism, aggregate properties are vital.}
Even if profiling instrumentation could provide perfect attribution of costs to executable instructions performance is often less dependent on the properties of individual source lines, and more a function of the data dependencies and balance among the statements in larger program units such as loops or loop nests.
\textbf{HPCToolkit}'s loop-level aggregation is particularly helpful.

\item \textbf{Measurement and analysis must be scalable.}
Scientific applications are increasingly executed on clusters with SMP nodes of multicore chips, creating several layers parallelism. 
It is critical that measurement techniques must be scalable to 10s and even 100s of thousands of threads. 

\end{itemize}

\textbf{HPCToolkit}'s website contains papers that explain these design principles in more detail.

\end{Name}

%%%%%%%%%%%%%%%%%%%%%%%%%%%%%%%%%%%%%%%%%%%%%%%%%%%%%%%%%%%%%%%%%%
\section{Description}

\subsection{Top-Down Analysis With Calling Context and Program Structure}

A typical node-based performance analysis session consists of:
\begin{enumerate}
\item \textbf{Measurement.}  
Collect low-overhead, high-accuracy profiles using statistical sampling.
\HTMLhref{hpcrun.html}{\Cmd{hpcrun}{1}} collects `flat' profiles (IP histograms, where IP is instruction pointer) while \Cmd{csprof}{1} collects call path profiles.

\item \textbf{Recovering static source code structure.} 
\HTMLhref{hpcstruct.html}{\Cmd{hpcstruct}{1}} recovers static program structure such as procedures and loop nests.
It accounts for procedure and loop transformations such as inlining and software pipelining.

\item \textbf{Correlating dynamic profiles with static source code structure.} 
\textbf{HPCToolkit}'s correlation tool combines dynamic profile information with \HTMLhref{hpcstruct.html}{\Cmd{hpcstruct}{1}}'s static program structure to correlate costs of the optimized object code to useful source code constructs such as loop nests.
The result of correlation is called an Experiment database.
Currently, \HTMLhref{hpcprof-flat.html}{\Cmd{hpcprof-flat}{1}} is used for flat profiles and \Cmd{hpcprof}{1} for call stack profiles.

\item \textbf{Top-down visualization}
\Prog{hpcviewer} is a Java-based top-down viewer for flat and call stack profiles.
An important feature of the Experiment database is that it is relocatable, containing profile information and copies of application source files.
This means that the first three steps can be performed remotely on a cluster and then the database can be viewed locally on a laptop or workstation.
\end{enumerate}

\subsection{Textual Analysis}

\begin{enumerate}
\item \textbf{Measurement.}  
Collect low-overhead, high-accuracy `flat' profiles using statistical sampling (\HTMLhref{hpcrun.html}{\Cmd{hpcrun}{1}}).

\item \textbf{Correlating dynamic profiles with procedures source lines.} 
\HTMLhref{hpcproftt.html}{\Cmd{hpcproftt}{1}} correlates `flat' profiles with source structure and produces textual output.

\end{enumerate}

%%%%%%%%%%%%%%%%%%%%%%%%%%%%%%%%%%%%%%%%%%%%%%%%%%%%%%%%%%%%%%%%%%
\section{Examples}

\subsection{Visualize native metrics from flat profiles}

Assume we have an application called \emph{zoo} whose source code is located in in \File{path-to-zoo}.

\begin{enumerate}
\item \textbf{Measure}.  Profile with \HTMLhref{hpcrun.html}{\Cmd{hpcrun}{1}}.  Assume we wish to use two different sets of events and that the results are placed in \File{profile-file-1} and \File{profile-file-2}.
\begin{verbatim}
  hpcrun <event-set-1> zoo
  hpcrun <event-set-2> zoo
\end{verbatim}

\item \textbf{Recover static source code structure}. Use \HTMLhref{hpcstruct.html}{\Cmd{hpcstruct}{1}} to recover program structure.
\begin{verbatim}
  hpcstruct zoo > zoo.psxml
\end{verbatim}

\item \textbf{Correlate dynamic profiles with static source code structure}.
Create an Experiment database using \HTMLhref{hpcprof-flat.html}{\Cmd{hpcprof-flat}{1}}.  Assume the Experiment database is located in \File{experiment-db}.
\begin{verbatim}
  hpcprof-flat -I path-to-zoo/\* -S zoo.psxml profile-file-1 profile-file-2
\end{verbatim}

\item \textbf{Visualize}.
Visualize the Experiment database using \HTMLhref{hpcviewer.html}{\Cmd{hpcviewer}{1}}:
\begin{verbatim}
  hpcviewer experiment-db/experiment.xml
\end{verbatim}

\end{enumerate}

\subsection{Visualize derived metrics for flat profiles}

The easiest way to begin is to collect all native metrics and then perform steps 1, 2 and 3 above.
These steps create a template \HTMLhref{hpcprof-flat.html}{\Cmd{hpcprof-flat}{1}} configuration file which can be modified and used to rerun \Prog{hpcprof-flat}.

\begin{enumerate}
\item \textbf{Create derived metrics}.  Add COMPUTE metrics to the \HTMLhref{hpcprof-flat.html}{\Cmd{hpcprof-flat}{1}} configuration file (named \File{config.xml}) generated during the initial run.  The \HTMLhref{hpcprof-flat.html}{\Cmd{hpcprof-flat}{1}} man page contains some examples. 

\item \textbf{Correlate}. Rerun \HTMLhref{hpcprof-flat.html}{\Cmd{hpcprof-flat}{1}} to create a new Experiment database.  Assume that it is located in \File{experiment-db-1}
\begin{verbatim}
  hpcprof-flat --config config.xml
\end{verbatim}

\item \textbf{Visualize}.
Visualize the Experiment database using \HTMLhref{hpcviewer.html}{\Cmd{hpcviewer}{1}}:
\begin{verbatim}
  hpcviewer experiment-db-1/experiment.xml
\end{verbatim}

\end{enumerate}


%%%%%%%%%%%%%%%%%%%%%%%%%%%%%%%%%%%%%%%%%%%%%%%%%%%%%%%%%%%%%%%%%%
%\section{Notes}

%%%%%%%%%%%%%%%%%%%%%%%%%%%%%%%%%%%%%%%%%%%%%%%%%%%%%%%%%%%%%%%%%%
\section{See Also}

\HTMLhref{hpcrun.html}{\Cmd{hpcrun}{1}}, \Cmd{csprof}{1} \\
\HTMLhref{hpcstruct.html}{\Cmd{hpcstruct}{1}} \\
\HTMLhref{hpcprof-flat.html}{\Cmd{hpcprof-flat}{1}}, \HTMLhref{hpcprof.html}{\Cmd{hpcprof}{1}}, \HTMLhref{hpcproftt.html}{\Cmd{hpcproftt}{1}} \\ %\Cmd{xprof}{1} 
\HTMLhref{hpcviewer.html}{\Cmd{hpcviewer}{1}}

%%%%%%%%%%%%%%%%%%%%%%%%%%%%%%%%%%%%%%%%%%%%%%%%%%%%%%%%%%%%%%%%%%
\section{Version}

Version: \Version\ of \Date.

%%%%%%%%%%%%%%%%%%%%%%%%%%%%%%%%%%%%%%%%%%%%%%%%%%%%%%%%%%%%%%%%%%
\section{License and Copyright}

\begin{description}
\item[Copyright] \copyright\ 2002-2007, Rice University.
\item[License] See \File{README.License}.
\end{description}

%%%%%%%%%%%%%%%%%%%%%%%%%%%%%%%%%%%%%%%%%%%%%%%%%%%%%%%%%%%%%%%%%%
\section{Authors}

Email: \Email{hpc@cs.rice.edu} \\
WWW: \URL{http://hipersoft.cs.rice.edu/hpctoolkit}.

\LatexManEnd

\end{document}

%% Local Variables:
%% eval: (add-hook 'write-file-hooks 'time-stamp)
%% time-stamp-start: "setDate{ "
%% time-stamp-format: "%:y/%02m/%02d"
%% time-stamp-end: "}\n"
%% time-stamp-line-limit: 50
%% End:

