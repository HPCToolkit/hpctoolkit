%% $Id$

%%%%%%%%%%%%%%%%%%%%%%%%%%%%%%%%%%%%%%%%%%%%%%%%%%%%%%%%%%%%%%%%%%%%%%%%%%%%%
%%%%%%%%%%%%%%%%%%%%%%%%%%%%%%%%%%%%%%%%%%%%%%%%%%%%%%%%%%%%%%%%%%%%%%%%%%%%%

\documentclass[english]{article}
\usepackage[latin1]{inputenc}
\usepackage{babel}
\usepackage{verbatim}

%% do we have the `hyperref package?
\IfFileExists{hyperref.sty}{
   \usepackage[bookmarksopen,bookmarksnumbered]{hyperref}
}{}

%% do we have the `fancyhdr' or `fancyheadings' package?
\IfFileExists{fancyhdr.sty}{
\usepackage[fancyhdr]{latex2man}
}{
\IfFileExists{fancyheadings.sty}{
\usepackage[fancy]{latex2man}
}{
\usepackage[nofancy]{latex2man}
\message{no fancyhdr or fancyheadings package present, discard it}
}}

%% do we have the `rcsinfo' package?
\IfFileExists{rcsinfo.sty}{
\usepackage[nofancy]{rcsinfo}
\rcsInfo $Id$
\setDate{\rcsInfoLongDate}
}{
\setDate{ 2007/05/28}
\message{package rcsinfo not present, discard it}
}

\setVersionWord{Version:}  %%% that's the default, no need to set it.
\setVersion{=PACKAGE_VERSION=}

%%%%%%%%%%%%%%%%%%%%%%%%%%%%%%%%%%%%%%%%%%%%%%%%%%%%%%%%%%%%%%%%%%%%%%%%%%%%%
%%%%%%%%%%%%%%%%%%%%%%%%%%%%%%%%%%%%%%%%%%%%%%%%%%%%%%%%%%%%%%%%%%%%%%%%%%%%%

\begin{document}

\begin{Name}{1}{hpcquick}{The HPCToolkit Performance Tools}{The HPCToolkit Performance Tools}{hpcquick:\\ A Front-end for hpcview}

\Prog{hpcquick} is a front-end for \Cmd{hpcview}{1}.
It creates an \Cmd{hpcview}{1} configuration file and optionally runs \Prog{hpcview} to create an Experiment database.

\end{Name}

%%%%%%%%%%%%%%%%%%%%%%%%%%%%%%%%%%%%%%%%%%%%%%%%%%%%%%%%%%%%%%%%%%
\section{Synopsis}

\Prog{hpcuick} \oOpt{other-options}
               \oOptArg{-I}{dir}
               \oOptArg{-S}{struct-file}
%              \oOptArg{-G}{group-file}
               \OptArg{-P}{profile-file...}

%%%%%%%%%%%%%%%%%%%%%%%%%%%%%%%%%%%%%%%%%%%%%%%%%%%%%%%%%%%%%%%%%%
\section{Description}

\Prog{hpcquick} is a front-end for \Cmd{hpcview}{1}.
It creates an \Cmd{hpcview}{1} configuration file (\File{./hpcquick.xml}) and runs \Prog{hpcview} to create an Experiment database.
All shell commands used to create the Experiment database are recorded in a log file (\File{./hpcquick.log}).

\Prog{hpcquick} expects a list of profile files (typically from \Cmd{hpcrun}{1}).
For best results, two other options should be used: \textbf{-I} to provide paths for source code directories and \textbf{-S} to provide source code structure from \Cmd{bloop}{1}.


%%%%%%%%%%%%%%%%%%%%%%%%%%%%%%%%%%%%%%%%%%%%%%%%%%%%%%%%%%%%%%%%%%
\section{Arguments}

\subsection{Typical options and arguments}

\begin{Description}
\item[\OptArg{-I}{dir}] A directory (relative or absolute) containing source code to which performance data should be correlated.  In order to search a directory \emph{and} recursively search all of its descendents, append an escaped `*' after the last slash, e.g., \verb+/mypath/\*+ (escaping is for the shell). May be passed multiple times.
\item[\OptArg{-S}{struct-file}] A structure file (from \Cmd{bloop}{1}) for the main program and/or any or all of the shared libraries used by the program.  May be passed multiple times.
%\item[\OptArg{-G}{group-file}] A group file for the main program and/or any or all of the shared libraries used by the program. May be passed multiple times.
\item[\OptArg{-P}{profile-file}] A space-separated list of profile files, which may be of the following types:
\begin{itemize}
  \item binary output from \Cmd{hpcrun}{1} (or \Cmd{hpcex}{1})
  \item binary output from SGI's ssrun and 
  \item XML PROFILE files (e.g., from \Cmd{hpcprof}{1} and \Prog{xprof})
\end{itemize}

\end{Description}

\subsection{Options: General}

\begin{Description}
\item[\Opt{-n}] No \Cmd{hpcview}{1} output.  Convert profiles and create a configuration file, but do not generate an Experiment database by running \Cmd{hpcview}{1}.
\item[\Opt{-V}] Print version information.
\item[\Opt{-h}] Print help.
\end{Description}

%%%%%%%%%%%%%%%%%%%%%%%%%%%%%%%%%%%%%%%%%%%%%%%%%%%%%%%%%%%%%%%%%%
\section{Examples}

%\begin{enumerate}
%\item 
[FIXME]

%\end{enumerate}

%%%%%%%%%%%%%%%%%%%%%%%%%%%%%%%%%%%%%%%%%%%%%%%%%%%%%%%%%%%%%%%%%%
%\section{Notes}


%%%%%%%%%%%%%%%%%%%%%%%%%%%%%%%%%%%%%%%%%%%%%%%%%%%%%%%%%%%%%%%%%%
\section{See Also}

\Cmd{hpctoolkit}{1}.

%%%%%%%%%%%%%%%%%%%%%%%%%%%%%%%%%%%%%%%%%%%%%%%%%%%%%%%%%%%%%%%%%%
\section{Version}

Version: \Version\ of \Date.

%%%%%%%%%%%%%%%%%%%%%%%%%%%%%%%%%%%%%%%%%%%%%%%%%%%%%%%%%%%%%%%%%%
\section{License and Copyright}

\begin{description}
\item[Copyright] \copyright\ 2002-2007, Rice University.
\item[License] See \File{README.License}.
\end{description}

%%%%%%%%%%%%%%%%%%%%%%%%%%%%%%%%%%%%%%%%%%%%%%%%%%%%%%%%%%%%%%%%%%
\section{Authors}

\noindent
John Mellor-Crummey \\
Nathan Tallent \\
Rob Fowler \\
Email: \Email{hpc@cs.rice.edu} \\
WWW: \URL{http://hipersoft.cs.rice.edu/hpctoolkit}.

\LatexManEnd

\end{document}

%% Local Variables:
%% eval: (add-hook 'write-file-hooks 'time-stamp)
%% time-stamp-start: "setDate{ "
%% time-stamp-format: "%:y/%02m/%02d"
%% time-stamp-end: "}\n"
%% time-stamp-line-limit: 50
%% End:

