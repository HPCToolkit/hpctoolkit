%% $Id$

%%%%%%%%%%%%%%%%%%%%%%%%%%%%%%%%%%%%%%%%%%%%%%%%%%%%%%%%%%%%%%%%%%%%%%%%%%%%%
%%%%%%%%%%%%%%%%%%%%%%%%%%%%%%%%%%%%%%%%%%%%%%%%%%%%%%%%%%%%%%%%%%%%%%%%%%%%%

\documentclass[english]{article}
\usepackage[latin1]{inputenc}
\usepackage{babel}
\usepackage{verbatim}

%% do we have the `hyperref package?
\IfFileExists{hyperref.sty}{
   \usepackage[bookmarksopen,bookmarksnumbered]{hyperref}
}{}

%% do we have the `fancyhdr' or `fancyheadings' package?
\IfFileExists{fancyhdr.sty}{
\usepackage[fancyhdr]{latex2man}
}{
\IfFileExists{fancyheadings.sty}{
\usepackage[fancy]{latex2man}
}{
\usepackage[nofancy]{latex2man}
\message{no fancyhdr or fancyheadings package present, discard it}
}}

%% do we have the `rcsinfo' package?
\IfFileExists{rcsinfo.sty}{
\usepackage[nofancy]{rcsinfo}
\rcsInfo $Id$
\setDate{\rcsInfoLongDate}
}{
\setDate{ 2008/07/29}
\message{package rcsinfo not present, discard it}
}

\setVersionWord{Version:}  %%% that's the default, no need to set it.
\setVersion{=PACKAGE_VERSION=}

%%%%%%%%%%%%%%%%%%%%%%%%%%%%%%%%%%%%%%%%%%%%%%%%%%%%%%%%%%%%%%%%%%%%%%%%%%%%%
%%%%%%%%%%%%%%%%%%%%%%%%%%%%%%%%%%%%%%%%%%%%%%%%%%%%%%%%%%%%%%%%%%%%%%%%%%%%%

\begin{document}

\begin{Name}{1}{hpcproftt}{The HPCToolkit Performance Tools}{The HPCToolkit Performance Tools}{hpcproftt:\\ Correlation of Flat Profile Metrics for Teletype Output}

\Prog{hpcproftt} correlates `flat' profile metrics with either source code structure or object code and generates textual output suitable for a terminal.
Alternatively, it also generates textual dumps of profile files.

\end{Name}

%%%%%%%%%%%%%%%%%%%%%%%%%%%%%%%%%%%%%%%%%%%%%%%%%%%%%%%%%%%%%%%%%%
\section{Synopsis}

\Prog{hpcproftt} [--source] \oOpt{options} \Arg{profile-file}...

\Prog{hpcproftt} --object \oOpt{options} \Arg{profile-file}...

\Prog{hpcproftt} --dump \Arg{profile-file}...


%%%%%%%%%%%%%%%%%%%%%%%%%%%%%%%%%%%%%%%%%%%%%%%%%%%%%%%%%%%%%%%%%%
\section{Description}

\Prog{hpcproftt} correlates `flat' profile metrics with either source code structure (the first and default mode) or object code (second mode) and generates textual output suitable for a terminal (hence the tt, for teletype, in its name).
In both of these modes, \Prog{hpcproftt} expects a list of flat profile files.
\Prog{hpcproftt} also supports a third mode in which it generates textual dumps of profile files.
In this mode, the profile list may contain either flat or call path profile files.

\Prog{hpcproftt} defaults to source code correlation mode.
Without any mode switch, it behaves as if passed \Opt{--source=pgm,lm}.

\textbf{Source Source Structure Correlation}.
First, \Prog{hpcproftt} creates raw metrics for every native event with in each profile file.
Derived summary metrics can be optionally created with the \Opt{--metric} option.
All metrics are normalized to use the unit 'events' rather than 'samples' since this enables meaningful comparisons and derived metrics.
Since percents facilitate rapid comprehension (compared to values), all raw metrics display as percents; derived metrics likewise default to percent, if sensible.

Then, \Prog{hpcproftt} correlates metrics to program structure.
If \HTMLhref{hpcstruct.html}{\Cmd{hpcstruct}{1}} structure information is available, this is used for correlation; if not, a simple 'File | Procedure | Statement' structure is computed using the load module's line map.

Finally, \Prog{hpcproftt} generates metric summaries and annotated source files to standard out.
Each metric summary compares a certain source structure element (such as a procedure) with all other elements of that type across the whole program.
Structure elements include: Program, Load Module, File, Procedure, Loop (requires \HTMLhref{hpcstruct.html}{\Cmd{hpcstruct}{1}}), and Statement.
For example, the procedure summary shows (exclusive) metric values for each procedure in the program.
(Structure elements are pruned if all corresponding metrics are 0.)
Summaries are rank ordered by the first metric.
\Prog{hpcproftt} optionally annotates source files with Statement (line) level metrics.
Note that it can only annotate files found by combining debug information with \Opt{--include} search paths.

%%% If the same native event name appears more than once, min/max/sum \emph{derived} events are computed.

\textbf{Object Correlation}.

\Prog{hpcproftt}'s object correlation mode is intended for inspection of fine-grained correlation.
In contrast to the source structure correlation mode, true summaries are not computed; rather (like an annotated source file) \Prog{hpcproftt} generates annotated object code, i.e., procedures and instructions.
Moreover, only raw metrics corresponding to the native events in \textbf{one} profile file may be correlated to the object code touched by those metrics.
Accordingly, \Prog{hpcproftt} creates raw metrics for every native event in one profile file.
Metrics use the unit 'samples' rather than 'events' and default to percents, though absolute values may optionally be displayed.
Procedures are pruned from the output if no associated metric totals to the provided threshold.

Warnings:
\begin{itemize}
\item On ISA's with variable sized instructions, histogram buckets (4 bytes in size) may contain information for more than one instruction.
In this case, multiple instructions will report counts for the \emph{same} bucket.

\item On some architectures, delays between event triggers, interrupt generation and sampling of the IP means that an event may be associated with a different instruction than one that caused the event.  It is not abnormal for this gap to be 50-70 instructions wide.
\end{itemize}

%%%%%%%%%%%%%%%%%%%%%%%%%%%%%%%%%%%%%%%%%%%%%%%%%%%%%%%%%%%%%%%%%%
\section{Arguments}

\begin{Description}
\item[\Arg{profile-file}...] A list of flat profile files.
\end{Description}

Default values for an option's optional arguments are shown in \{\}.

\subsection{Options: General}

\begin{Description}
\item[\OptoArg{-v}{n}, \OptoArg{--verbose}{n}] Verbose: generate progress messages to stderr at verbosity level \Arg{n}.  \{1\} 
\item[\Opt{-V}, \Opt{--version}] Print version information.
\item[\Opt{-h}, \Opt{--help}] Print help.
\item[\OptoArg{--debug}{n}]   Debug: use debug level \Arg{n}. \{1\}
\end{Description}

\subsection{Options: Source Structure Correlation}

\begin{Description}
\item[\OptoArg{--source}{=all,sum,pgm,lm,f,p,l,s,src}]
\item[\OptoArg{--src}{=all,sum,pgm,lm,f,p,l,s,src}] 
Correlate metrics to source code structure. Without \Opt{--source}, default is \{pgm,lm\}; with, it is \{sum\}.
  \begin{itemize}
  \item all: all summaries plus annotated source files
  \item sum: all summaries
  \item pgm: program summary
  \item lm:  load module summary
  \item f:   files summary
  \item p:   procedure summary
  \item l:   loop summary
  \item s:   statement summary
  \item src: annotate source files with metrics; equivalent to \Opt{--srcannot '*'}
  \end{itemize}

\item[\OptArg{--srcannot}{glob}] 
Annotate source files with path names that match file glob \Arg{glob}. (Protect glob characters from shell with single quotes or backslash.) May pass multiple times to logically OR additional globs.

\item[\OptoArg{-M}{metric}, \OptoArg{--metric}{metric}] 
Show a supplemental or different metric set. \Arg{metric} is one of the following:
  \begin{itemize}
  \item sum:      Show also Mean, CoefVar (coefficient of variation), Min, Max, Sum
  \item sum-only: Show only Mean, CoefVar (coefficient of variation), Min, Max, Sum
  \end{itemize}

\item[\OptArg{-I}{path}, \OptArg{--include}{path}] 
Use \Arg{path} when searching for source files. For a recursive search, append a '*' after the last slash, e.g., \verb+'/mypath/*'+ (quote or escape to protect from the shell). May pass multiple times.

\item[\OptArg{-S}{file}, \OptArg{--structure}{file}] 
Use \HTMLhref{hpcstruct.html}{\Cmd{hpcstruct}{1}} structure file \Arg{file} for correlation.  May pass multiple times (e.g., for shared libraries).
\end{Description}

\subsection{Options: Object Correlation}

\begin{Description}

\item[\OptoArg{--object}{=s}] 
\item[\OptoArg{--obj}{=s}] 
Correlate metrics with object code by annotating object code procedures and instructions. \{\}
  \begin{itemize}
  \item s: intermingle source line info with object code
  \end{itemize}

\item[\OptArg{--objannot}{glob}] 
Annotate object procedures with (unmangled) names that match glob \Arg{glob}. (Protect glob characters from shell with single quotes or backslash.) May pass multiple times to logically OR additional globs.

\item[\OptArg{--obj-threshold}{n}] 
Prune procedures with an event count < \Arg{n} \{1\}

\item[\Opt{--obj-values}] 
Show raw metrics as values instead of percents

\end{Description}

\subsection{Options: Dump Raw Profile Data}

\begin{Description}
\item[\Opt{--dump}] Generate textual representation of raw profile data.
\end{Description}


%%%%%%%%%%%%%%%%%%%%%%%%%%%%%%%%%%%%%%%%%%%%%%%%%%%%%%%%%%%%%%%%%%
\section{Examples}

\begin{itemize}

\item Assume we have collected flat profiling information using \HTMLhref{hpcrun-flat.html}{\Cmd{hpcrun-flat}{1}}).
Let the executable and profile be named \File{poissonSolve} and \File{poissonSolve.PAPI_L2_DCM-etc...}, respectively.
Assume further that we wish to correlate derived summary metrics (Mean, RStdDev, Min, Max) with source code and generate summary tables.
To do this, execute:
\begin{verbatim}
    hpcproftt --src --metric=sum-only poissonSolve.hpcrun*
\end{verbatim}

\item Using the same example as above, assume we wish to correlate the profile with object code.  Execute the following command:
\begin{verbatim}
    hpcproftt --obj poissonSolve.hpcrun*
\end{verbatim}

\end{itemize}

%%%%%%%%%%%%%%%%%%%%%%%%%%%%%%%%%%%%%%%%%%%%%%%%%%%%%%%%%%%%%%%%%%
%\section{Notes}

%%%%%%%%%%%%%%%%%%%%%%%%%%%%%%%%%%%%%%%%%%%%%%%%%%%%%%%%%%%%%%%%%%
\section{See Also}

\HTMLhref{hpctoolkit.html}{\Cmd{hpctoolkit}{1}}.

%%%%%%%%%%%%%%%%%%%%%%%%%%%%%%%%%%%%%%%%%%%%%%%%%%%%%%%%%%%%%%%%%%
\section{Version}

Version: \Version\ of \Date.

%%%%%%%%%%%%%%%%%%%%%%%%%%%%%%%%%%%%%%%%%%%%%%%%%%%%%%%%%%%%%%%%%%
\section{License and Copyright}

\begin{description}
\item[Copyright] \copyright\ 2002-2008, Rice University.
\item[License] See \File{README.License}.
\end{description}

%%%%%%%%%%%%%%%%%%%%%%%%%%%%%%%%%%%%%%%%%%%%%%%%%%%%%%%%%%%%%%%%%%
\section{Authors}

\noindent
Nathan Tallent \\
John Mellor-Crummey \\
Rob Fowler \\
Email: \Email{hpc@cs.rice.edu} \\
WWW: \URL{http://hipersoft.cs.rice.edu/hpctoolkit}.

\LatexManEnd

\end{document}

%% Local Variables:
%% eval: (add-hook 'write-file-hooks 'time-stamp)
%% time-stamp-start: "setDate{ "
%% time-stamp-format: "%:y/%02m/%02d"
%% time-stamp-end: "}\n"
%% time-stamp-line-limit: 50
%% End:

