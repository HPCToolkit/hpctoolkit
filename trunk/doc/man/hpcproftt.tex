%% $Id$

%%%%%%%%%%%%%%%%%%%%%%%%%%%%%%%%%%%%%%%%%%%%%%%%%%%%%%%%%%%%%%%%%%%%%%%%%%%%%
%%%%%%%%%%%%%%%%%%%%%%%%%%%%%%%%%%%%%%%%%%%%%%%%%%%%%%%%%%%%%%%%%%%%%%%%%%%%%

\documentclass[english]{article}
\usepackage[latin1]{inputenc}
\usepackage{babel}
\usepackage{verbatim}

%% do we have the `hyperref package?
\IfFileExists{hyperref.sty}{
   \usepackage[bookmarksopen,bookmarksnumbered]{hyperref}
}{}

%% do we have the `fancyhdr' or `fancyheadings' package?
\IfFileExists{fancyhdr.sty}{
\usepackage[fancyhdr]{latex2man}
}{
\IfFileExists{fancyheadings.sty}{
\usepackage[fancy]{latex2man}
}{
\usepackage[nofancy]{latex2man}
\message{no fancyhdr or fancyheadings package present, discard it}
}}

%% do we have the `rcsinfo' package?
\IfFileExists{rcsinfo.sty}{
\usepackage[nofancy]{rcsinfo}
\rcsInfo $Id$
\setDate{\rcsInfoLongDate}
}{
\setDate{ 2008/05/28}
\message{package rcsinfo not present, discard it}
}

\setVersionWord{Version:}  %%% that's the default, no need to set it.
\setVersion{=PACKAGE_VERSION=}

%%%%%%%%%%%%%%%%%%%%%%%%%%%%%%%%%%%%%%%%%%%%%%%%%%%%%%%%%%%%%%%%%%%%%%%%%%%%%
%%%%%%%%%%%%%%%%%%%%%%%%%%%%%%%%%%%%%%%%%%%%%%%%%%%%%%%%%%%%%%%%%%%%%%%%%%%%%

\begin{document}

\begin{Name}{1}{hpcproftt}{The HPCToolkit Performance Tools}{The HPCToolkit Performance Tools}{hpcproftt:\\ Correlation of Flat Profile Metrics for Teletype Output}

\Prog{hpcproftt} correlates `flat' profile metrics with either source code structure or object code and generates textual output suitable for a terminal.
Alternatively, it also generates textual dumps of profile files.

\end{Name}

%%%%%%%%%%%%%%%%%%%%%%%%%%%%%%%%%%%%%%%%%%%%%%%%%%%%%%%%%%%%%%%%%%
\section{Synopsis}

\Prog{hpcproftt} [--source] \oOpt{options} \Arg{profile-file}...

\Prog{hpcproftt} --object \oOpt{options} \Arg{profile-file}...

\Prog{hpcproftt} --dump \Arg{profile-file}...


%%%%%%%%%%%%%%%%%%%%%%%%%%%%%%%%%%%%%%%%%%%%%%%%%%%%%%%%%%%%%%%%%%
\section{Description}

\Prog{hpcproftt} correlates `flat' profile metrics with either source code structure (first mode) or object code (second mode) and generates textual output suitable for a terminal (hence the tt, for teletype, in its name).
In each of these modes, \Prog{hpcproftt} expects a list of flat profile files 

\Prog{hpcproftt} also supports a third mode in which it generates textual dumps of profile files.
In this mode, the profile list may contain either flat or call path profile files.


\Prog{hpcproftt} can generate the following information.  Note that the term \emph{native} event refers to an event for which there is a program counter histogram.  
\begin{Description}
  \item[totals] Total sample counts for each native event, over each load module.  If the same native event name appears more than once, min/max/sum \emph{derived} events are computed.
  \item[load-module correlation] For each \emph{native} event (in order), show the exclusive percentage (or number) of samples attributed to each load module in the \Arg{hpcrun-file}s.  
Percentages are relative to totals.
  \item[file correlation] For each \emph{native} event (in order), show the exclusive percentage (or number) of samples attributed to each $<$load module$>$$<$source file$>$ pair.  
Percentages are relative to totals.
  \item[function correlation] For each \emph{native} event (in order), show the exclusive percentage (or number) of samples attributed to each $<$load module$>$$<$function$>$ pair.
Percentages are relative to totals.
  \item[line correlation] For each \emph{native} event (in order), show the exclusive percentage (or number) of samples attributed to each $<$load module$>$$<$file$>$$<$line$>$ triple.
Percentages are relative to totals.
  \item[annotated source files] For each $<$load module$>$$<$source file$>$ pair, annotate the file's source lines with line correlation information.
  \item[object code correlation] For each \emph{native} event (in order), show the exclusive percentage (or number) of samples attributed to each (text segment) object code instruction from each load module.
Percentages annotating an instruction are relative to that procedure.
By default, only procedures where some event has a count greater than 0 are shown.
NOTE: On ISA's with variable sized instructions, histogram buckets (4 bytes in size) may contain information for more than one instruction.  In this case, multiple instructions will report counts for the \emph{same} bucket.
\end{Description}

%%%%%%%%%%%%%%%%%%%%%%%%%%%%%%%%%%%%%%%%%%%%%%%%%%%%%%%%%%%%%%%%%%
\section{Arguments}

\begin{Description}
\item[\Arg{profile-file}...] A list of flat profile files.
\end{Description}

Default values for an option's optional arguments are shown in \{\}.

\subsection{Options: General}

\begin{Description}
\item[\OptoArg{-v}{n}, \OptoArg{--verbose}{n}] Verbose: generate progress messages to stderr at verbosity level \Arg{n}.  \{1\} 
\item[\Opt{-V}, \Opt{--version}] Print version information.
\item[\Opt{-h}, \Opt{--help}] Print help.
\item[\OptoArg{--debug}{n}]   Debug: use debug level \Arg{n}. \{1\}
\end{Description}

\subsection{Options: Source Structure Correlation}

\begin{Description}
\item[\OptArg{-I}{path}, \OptArg{--include}{path}] Use \Arg{path} when searching for source files. To recursively search a path, append an escaped `*' after the last slash, e.g., \verb+/mypath/\*+ (escaping is for the shell). May pass multiple times.
\item[\OptArg{-S}{file}, \OptArg{--structure}{file}] Use \HTMLhref{hpcstruct.html}{\Cmd{hpcstruct}{1}} structure file \Arg{file} for correlation.  May pass multiple times (e.g., for shared libraries).
\end{Description}

\subsection{Options: Object Correlation}

\begin{Description}
\item[\OptArg{--object}{...}] Show object code correlation: load modules, procedures, instructions...
\end{Description}

\subsection{Options: Dump Raw Profile Data}

\begin{Description}
\item[\Opt{--dump}] Generate textual representation of raw profile data.
\end{Description}


%%%%%%%%%%%%%%%%%%%%%%%%%%%%%%%%%%%%%%%%%%%%%%%%%%%%%%%%%%%%%%%%%%
\section{Examples}

\begin{itemize}

\item Assume we have collected flat profiling information using \HTMLhref{hpcrun.html}{\Cmd{hpcrun}{1}} \Cmd{hpcex}{1}).
Let the executable and profile be named \File{poissonSolve} and \File{poissonSolve.PAPI_L2_DCM-etc...}, respectively.
Assume further that we wish to correlate the profile with source lines.
To do this, execute:
\begin{verbatim}
    hpcproftt -l poissonSolve sweep3dsingle poissonSolve.PAPI_L2_DCM-etc...
\end{verbatim}

\item Using the same example as above, assume we wish to correlate the profile with object code.  Execute the following command:
\begin{verbatim}
    hpcproftt -o poissonSolve sweep3dsingle poissonSolve.PAPI_L2_DCM-etc...
\end{verbatim}

\end{itemize}

%%%%%%%%%%%%%%%%%%%%%%%%%%%%%%%%%%%%%%%%%%%%%%%%%%%%%%%%%%%%%%%%%%
%\section{Notes}

%%%%%%%%%%%%%%%%%%%%%%%%%%%%%%%%%%%%%%%%%%%%%%%%%%%%%%%%%%%%%%%%%%
\section{See Also}

\HTMLhref{hpctoolkit.html}{\Cmd{hpctoolkit}{1}}.

%%%%%%%%%%%%%%%%%%%%%%%%%%%%%%%%%%%%%%%%%%%%%%%%%%%%%%%%%%%%%%%%%%
\section{Version}

Version: \Version\ of \Date.

%%%%%%%%%%%%%%%%%%%%%%%%%%%%%%%%%%%%%%%%%%%%%%%%%%%%%%%%%%%%%%%%%%
\section{License and Copyright}

\begin{description}
\item[Copyright] \copyright\ 2002-2007, Rice University.
\item[License] See \File{README.License}.
\end{description}

%%%%%%%%%%%%%%%%%%%%%%%%%%%%%%%%%%%%%%%%%%%%%%%%%%%%%%%%%%%%%%%%%%
\section{Authors}

\noindent
Nathan Tallent \\
John Mellor-Crummey \\
Rob Fowler \\
Email: \Email{hpc@cs.rice.edu} \\
WWW: \URL{http://hipersoft.cs.rice.edu/hpctoolkit}.

\LatexManEnd

\end{document}

%% Local Variables:
%% eval: (add-hook 'write-file-hooks 'time-stamp)
%% time-stamp-start: "setDate{ "
%% time-stamp-format: "%:y/%02m/%02d"
%% time-stamp-end: "}\n"
%% time-stamp-line-limit: 50
%% End:

