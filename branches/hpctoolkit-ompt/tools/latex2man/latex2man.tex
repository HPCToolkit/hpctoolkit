%%%%%%%%%%%%%%%%%%%%%%%%%%%%%%%%%%%%%%%%%%%%%%%%%%%%%%%%%%%%%%%%%%%%%%%%%%%%%%%
%% Project:     Documentation Tools
%% Descr:       Latex -->  MAN-page (groff -man), LATEX documentation
%% Author:      Dr. J�rgen Vollmer, Juergen.Vollmer@informatik-vollmer.de
%% $Id: latex2man.tex,v 1.187 2008/11/11 07:21:18 vollmer Exp $
%%%%%%%%%%%%%%%%%%%%%%%%%%%%%%%%%%%%%%%%%%%%%%%%%%%%%%%%%%%%%%%%%%%%%%%%%%%%%%%
%%
%% Latex2man is a tool to translate UNIX manual pages written with LaTeX into a
%% format understood by the UNIX man(1)-command.
%% Alternatively HTML or TexInfo code can be produced too.
%% Output of parts of the text may be supressed using the conditional text
%% feature.
%%
%% There is LaTeX package (latex2man.sty) used for writing the Man-page
%% and a PERL script (latex2man) doing the actual translation.
%%
%% Copyright (C) 1998, Dr. Juergen Vollmer
%%                     Viktoriastrasse 15, D-76133 Karlsruhe, Germany
%%                     Juergen.Vollmer@informatik-vollmer.de
%% License:
%%   This program can be redistributed and/or modified under the terms
%%   of the LaTeX Project Public License Distributed from CTAN
%%   archives in directory macros/latex/base/lppl.txt; either
%%   version 1 of the License, or any later version.
%%
%% If you find this software useful, please send me a postcard.
%%
%%%%%%%%%%%%%%%%%%%%%%%%%%%%%%%%%%%%%%%%%%%%%%%%%%%%%%%%%%%%%%%%%%%%%%%%%%%%%%%

\documentclass[english]{article}
\usepackage[latin1]{inputenc}
\usepackage{babel}
\usepackage{verbatim}

%% do we have the `hyperref package?
\IfFileExists{hyperref.sty}{
   \usepackage[bookmarksopen,bookmarksnumbered]{hyperref}
}{}

%% do we have the `fancyhdr' package?
\IfFileExists{fancyhdr.sty}{
\usepackage[fancyhdr]{latex2man}
}{
%% do we have the `fancyheadings' package?
\IfFileExists{fancyheadings.sty}{
\usepackage[fancy]{latex2man}
}{
\usepackage[nofancy]{latex2man}
\message{no fancyhdr or fancyheadings package present, discard it}
}}

%% do we have the `rcsinfo' package?
\IfFileExists{rcsinfo.sty}{
\usepackage[nofancy]{rcsinfo}
\rcsInfo $Id: latex2man.tex,v 1.187 2008/11/11 07:21:18 vollmer Exp $
\setDate{\rcsInfoLongDate}
}{
\setDate{2008/11/11}    %%%% must be manually set, if rcsinfo is not present
\message{package rcsinfo not present, discard it}
}

\setVersionWord{Version:}  %%% that's the default, no need to set it.
\setVersion{1.23}

\begin{document}

\begin{Name}{1}{latex2man}{Dr.~J{\"u}rgen Vollmer}{Documentation Tools}{\LaTeX2man\\--\\ A Documentation Tool}

  \Prog{Latex2man} is a tool to translate UNIX manual pages written with
  \LaTeX\ into a format understood by the UNIX \Cmd{man}{1}-command.
  Alternatively HTML, TexInfo, or \LaTeX\ code can be produced too.
  Output of parts of the text may be suppressed using the conditional text
  feature (for this, LaTeX generation may be used).
\end{Name}

\section{Synopsis}
%%%%%%%%%%%%%%%%%%

\Prog{latex2man} \oOptArg{-t}{transfile}
                 \oOpt{-HMTL} \oOpt{-h} \oOpt{-V} \oOptArg{-C}{name} \TEXIbr
                 \oOptArg{-a}{char}
                 \Arg{infile}
                 \Arg{outfile}

\section{Description}
%%%%%%%%%%%%%%%%%%%%%
\Prog{Latex2man} reads the file \Arg{infile} and writes \Arg{outfile}. The
input must be a \LaTeX\ document using the \Prog{latex2man} \LaTeX\ package.
\Prog{Latex2man} translates that document into the \Cmd{troff}{1} format using
the \Opt{-man} macro package.

Using the \Opt{-H} option, HTML code can be produced, instead of
\Cmd{troff}{1}.

Using the \Opt{-T} option, TexInfo code can be produced, instead of
\Cmd{troff}{1}.

Using the \Opt{-M} option, \Cmd{troff}{1} input is produced.

Using the \Opt{-L} option, \LaTeX\ ouput can be produced, instead of
\Cmd{troff}{1}.


\section{Options}
%%%%%%%%%%%%%%%%%
\begin{Description}[\OptArg{-t}{transfile}]\setlength{\itemsep}{0cm}
\item[\OptArg{-t}{transfile}] Translation for user defined \LaTeX\ macros.
\item[\Opt{-M}] Produce output suitable for the \Cmd{man}{1} command (default).
\item[\Opt{-H}] Instead of producing output suitable for the \Cmd{man}{1}
     command, HTML code is produced (despite the name of the command).
\item[\Opt{-T}] Instead of producing output suitable for the \Cmd{man}{1}
     command, TexInfo code is produced (despite the name of the command).  The
     generated \File{.texi}-file may be processed with \Cmd{makeinfo}{1}
     (to produce an \File{.info}-file) which in turn may be installed using
     \Cmd{install-info}{1}.  The Info tags \verb+@dircategory+ and
     \verb+@direntry+ are provided.
\item[\Opt{-L}] The \LaTeX\ source is written to the \Arg{outfile}. This is
     useful in conjunction with the \OptArg{-C}{name} option.
\item[\OptArg{-C}{name}] Output the conditional text for \Arg{name}.  If more
     than one name should be given use quotes: \OptArg{-C}{'name1 name2 ...'}\\
     The following names are defined automatically:
     \begin{itemize}
        \item \Opt{-H} defines \texttt{HTML}
        \item \Opt{-T} defines \texttt{TEXI}
        \item \Opt{-M} defines \texttt{MAN}
        \item \Opt{-L} defines \texttt{LATEX}
     \end{itemize}
 \item[\OptArg{-a}{char}] Is used only in conjunction with -T.\\
     Background:\\
     TexInfo ignores all blanks before the first word on a new line. In order
     to produce some additional space before that word (using \Bs SP) some
     character has to be printed before the additional space. By default this
     is a . (dot).  The \Arg{char} specifies an alternative for that first
     character.  Giving a blank to
     \Opt{-a} supresses the indentation of a line.\\
     Note: only for the first \Bs SP of a series that \Arg{char} is printed.
\item[\Opt{-h}] Show a help text.
\item[\Opt{-V}] Show version information.
\end{Description}

\section{Files}
%%%%%%%%%%%%%%%

\begin{Description}\setlength{\itemsep}{0cm}
\item[\File{latex2man.tex}] The \LaTeX\ file containing this Man-page.
\item[\File{latex2man.sty}] The \LaTeX\ package defining the environments and
     commands.
\item[\File{latex2man.cfg}] The configuration file for \Prog{Latex2man}
     \LaTeX-package.
\item[\File{latex2man.trans}] File containing example translations of user
     defined \LaTeX\ macros.
\item[\File{fancyheadings.sty}] A \LaTeX\ package used to typeset head- and
     foot lines.
\item[\File{fancyhdr.sty}] A \LaTeX\ package used to typeset head- and foot
     lines.
\item[\File{rcsinfo.sty}] A \LaTeX\ package used to extract and use RCS version
     control information in \LaTeX\ documents.
\end{Description}

\section{See Also}
%%%%%%%%%%%%%%%%%%

\LaTeX, TexInfo, \Cmd{troff}{1}, \Cmd{groff}{1}, \Cmd{makeinfo}{1}.

\section{\LaTeX\ commands}
%%%%%%%%%%%%%%%%%%%%%%%%%%

The \LaTeX\ package \texttt{latex2man} is used to write the Man-pages with
\LaTeX.  Since we translate into other text formats, not all \LaTeX\ stuff can
be translated.

\subsection{Package Options}
%%%%%%%%%%%%%%%%%%%%%%%%%%%%
The \texttt{latex2man} package accepts the following options:

\begin{description}\setlength{\itemsep}{0cm}
\item[fancy]    use the  \LaTeX\ package \texttt{fancyheadings}.
\item[fancyhdr] use the  \LaTeX\ package \texttt{fancyhdr}.
\item[nofancy] neither the \LaTeX\ package \texttt{fancyheadings} nor
     \texttt{fancyhdr} are used.
\end{description}

The default option may be specified in the file \texttt{latex2man.cfg}.

\subsection{Package Specific Environments}
%%%%%%%%%%%%%%%%%%%%%%%%%%%%%%%%%%%%%%%%%%

The following environments are provided by the package:
\begin{description}\setlength{\itemsep}{0cm}
\item[\Bs begin\{Name\}\{chapter\}\{name\}\{author\}\{info\}\{title\} ] The
     \texttt{Name} environment takes five arguments: 1.\ the Man-page chapter,
     2.\ the name of the Man-page, 3.\ the author, 4.\ some short information
     about the tool printed in the footline of the Man-page, and 5.\ a text
     which is used as title, for HTML and \LaTeX\ (it's ignored for output of
     the Man-page or TeXinfo. The \texttt{Name} environment must be the first
     environment in the document. Processing starts with this environment. Any
     text before this is ignored (exception: the \verb+setVersion+ and
     \verb+setDate+ commands).  (Note: all arguments of \Bs begin\{Name\} must
     be written on one line).
\item[\Bs begin\{Table\}\Lbr width\Rbr\{columns\}] The \texttt{Table}
     environment takes two arguments: the first optional one specifies
     a width of the last column, the second one gives the number of columns.
     For example:

     \verb+\begin{Table}[2cm]{3}+            \\
     \verb+ Here & am  & I   \\\hline+  \\
     \verb+  A 1 & A 2 & A 3 1 2 3 4 5 A 3 1 2 3 4 5 \\+        \\
     \verb+  B 1 & B 2 & B 3 \\+        \\
     \verb+\end{Table}+

     will be typeset as:

     \begin{Table}[2cm]{3}
       Here & am  & I   \\\hline
        A 1 & A 2 & A 3 1 2 3 4 5 A 3 1 2 3 4 5\\
        B 1 & B 2 & B 3 \\
     \end{Table}

     If no optional \emph{width} argument is given, all entries are
     typeset left justified.
     The \emph{width} is a length measured absolutly in \emph{cm}.
     Processing with \LaTeX\ a \verb+p{width}+ column is typeset
     as last column. The translation to \Cmd{troff}{1} commands
     results in a \texttt{lw(width)} column specification. Translating
     to HTML and TexInfo ignores the \emph{width} parameter.

     \verb+\hline+ may be used.

     If the Man-page is formatted with \Cmd{troff}{1} and tables are used, the
     \Cmd{tbl}{1} preprocessor should be called, usually by giving
     a \Opt{-t} to the call of \Cmd{troff}{1}. When viewing the generated
     manula page using \Cmd{man}{1}, \Cmd{tbl}{1} is called automatically.
\item[\Bs begin\{Description\}] is the same as \Bs begin\{description\}
\item[\Bs begin\{Description\}\Lbr label\Rbr] is similar to
     \Bs begin\{description\}, but the item labels have at minimum the size
     of the (optional) word \emph{label}. The difference is visible only
     in the DVI and PDF-output, not in the troff, TexInfo or HTML output.
     \begin{description}
     \item[a]   \Bar a \verb+\begin{description}+
     \item[ab]  \Bar ab
     \item[abc] \Bar abc
     \end{description}
     \begin{Description}
     \item[a]  \Bar a \verb+\begin{Description}+
     \item[ab] \Bar ab
     \item[abc]\Bar abc
     \end{Description}
     \begin{Description}[ab]
     \item[a]   \Bar a \verb+\begin{Description}[aa]+
     \item[ab]  \Bar ab
     \item[abc] \Bar abc
     \end{Description}
\end{description}

\subsection{Accepted \LaTeX\ Environments}
%%%%%%%%%%%%%%%%%%%%%%%%%%%%%%%%%%%%%%%%%%

The following environments are accepted:
\begin{itemize}\setlength{\itemsep}{0cm}
\item \texttt{description}
\item \texttt{enumerate}
\item \texttt{itemize}
\item \texttt{verbatim}
\item \texttt{center}
\end{itemize}

They may be nested:
\begin{itemize}
\item Itemize and nested center:
     \begin{center}
       A centered line.\\
       Another centered line.
     \end{center}
\item Another item an nested enumerate
     \begin{enumerate}
       \item a
       \item b
     \end{enumerate}
\end{itemize}


\subsection{Package Specific Macros}
%%%%%%%%%%%%%%%%%%%%%%%%%%%%%%%%%%%%

The following commands are provided:
\begin{description}\setlength{\itemsep}{0cm}
\item[\Bs Opt\{option\}] Option:  \verb+\Opt{-o}+ will be typeset as \Opt{-o}.
\item[\Bs Arg\{argument\}] Argument: \verb+\Arg{filename}+ will be typeset as
     \Arg{filename}.
\item[\Bs OptArg\{option\}\{argument\}] Option with Argument:\\
     \verb+\OptArg{-o}{filename}+ will be typeset as \OptArg{-o}{filename}.
\item[\Bs OptoArg\{option\}\{argument\}] Option with optional Argument:\\
     \verb+\OptoArg{-o}{filename}+ will be
     typeset as \OptoArg{-o}{filename}.
\item[\Bs oOpt\{option\}] Optional option, e.g.\ \verb+\oOpt{-o}+ will be
     typeset as \oOpt{-o}.
\item[\Bs oArg\{argument\}] Optional argument, e.g.\ \verb+\oArg{filename}+
     will be typeset as \oArg{filename}.
\item[\Bs oOptArg\{option\}\{argument\}] Optional option with argument, e.g.\\
     \verb+\oOptArg{-o}{filename}+ will be typeset as \oOptArg{-o}{filename}.
\item[\Bs oOptoArg\{option\}\{argument\}] Optional option with optional
     argument, e.g.\
     \verb+\oOptoArg{-o}{filename}+ will be typeset as \oOptoArg{-o}{filename}.
\item[\Bs File\{filename\}] used to typeset filenames, e.g.\
     \verb+\File{filename}+ will be typeset as \File{filename}.
\item[\Bs Prog\{prog\}] used to typeset program names, e.g.\
     \verb+\Prog{latex2man}+ will be typeset as \Prog{latex2man}.
\item[\Bs Cmd\{command\}\{chapter\}] used to typeset references to other
     commands, e.g.\ \verb+\Cmd{latex2man}{1}+ will be typeset as
     \Cmd{latex2man}{1}.
\item[\Bs Bar] is typeset as \Bar.
\item[\Bs Bs] (BackSlash) is typeset as \Bs.
\item[\Bs Tilde] is typeset as a \Tilde.
\item[\Bs Dots] is typeset as \Dots
\item[\Bs Bullet] us typeset as \Bullet.
\item[\Bs setVersion\{..\}] set  \verb+..+ as version information.
\item[\Bs setVersionWord\{..\}] set \verb+..+ for the word \emph{Version:} in
     the footline. The default is \verb+\setVersionWord{Version:}+.
\item[\Bs Version] returns the version information.
\item[\Bs setDate\{..\}] sets \verb+..+ as date information.
\item[\Bs Date] returns the date information.
\item[\Bs Email\{..\}] use to mark an Email address:\\
        \verb+\Email{Juergen.Vollmer@informatik-vollmer.de}+ is typeset as:\\
        \Email{Juergen.Vollmer@informatik-vollmer.de}.
\item[\Bs URL\{..\}] use to mark an URL:
        \verb+\URL{http://www.foo.de/\Tilde vollmer}+ is typeset as
        \URL{http://www.foo.de/\Tilde vollmer}.
\item[\Bs LatexManEnd] the input file is read and processed until reading
        end-of-file or \verb+\LatexManEnd+ (at the beginning of a line).
        \LaTeX\ ignores this command.
\item[\Bs Lbr, \Bs Rbr] is typeset as \Lbr\ and \Rbr\ (these variants are
        needed only somtimes like in \verb+item[FooBar\LBr xx \Lbr]+. Usually
        \verb+[ ]+ will work.
\item[\Bs LBr, \Bs RBr] is typeset as \LBr\ and \RBr\ (these variants are
        needed when using \{ or \} as arguments to macros.
\item[\Bs Circum] is typeset as \Circum.
\item[\Bs Percent] is typeset as \Percent.
\item[\Bs TEXbr] If processed with \LaTeX\ causes a linebreak (i.e.\ is
     equivalent to \Bs\Bs). In the output of \Prog{latex2man} this macro is
     ignored.
\item[\Bs TEXIbr] If TexInfo output is generated, causes a linebreak (i.e.\ is
     equivalent to \Bs\Bs), otherwise ignored.
\item[\Bs MANbr] If Man-Page output is generated, causes a linebreak (i.e.\ is
     equivalent to \Bs\Bs), otherwise ignored.
\item[\Bs HTMLbr] If HTML output is generated, causes a linebreak (i.e.\ is
     equivalent to \Bs\Bs), otherwise ignored.
\item[\Bs medskip] An empty line.
\item[\Bs SP] Produces some extra space, works also at the beginning of lines.
     The code of the second line looks like:
     \texttt{\Bs SP abc \Bs SP\Bs SP xx\Bs\Bs}:\\
     abc  \SP\SP\SP xx\\
     \SP abc \SP\SP xx\\
     \SP\SP abc \SP xx

     Note: Due to some ``problems'' with TexInfo, the lines starting with
     \texttt{\Bs SP} have a leading . (dot) in the TexInfo output,
     see \OptArg{-a}{char}.
\end{description}

\subsection{Accepted Macros from the rcsinfo Package}
%%%%%%%%%%%%%%%%%%%%%%%%%%%%%%%%%%%%%%%%%%%%%%%%%%%%%

\begin{description}\setlength{\itemsep}{0cm}
\item[\Bs rcsInfo \$Id ...\$] if the \LaTeX\ package \File{rcsinfo} is used,
     this command is used to extract the date of the Man-page.
\item[\Bs rcsInfoLongDate] if the \LaTeX\ package \File{rcsinfo} is used, this
     command is used to typeset the date coded in the \verb+$Id ..$+ string.
\end{description}

\subsection{Accepted \LaTeX\ Macros}
%%%%%%%%%%%%%%%%%%%%%%%%%%%%%%%%%%%%

The following standard \LaTeX\ commands are accepted:
\begin{description}\setlength{\itemsep}{0cm}
\item[\Bs section\{..\}] The \texttt{section} macro takes one argument: the
     name of the Man-page section.  Each Man-page consists of several sections.
     Usually there are the following sections in a Man-page: \emph{Name}
     (special handling as environment, c.f.\ above), \emph{Synopsis},
     \emph{Description}, \emph{Options}, \emph{Files}, \emph{See Also},
     \emph{Diagnostics}, \emph{Return Values}, \emph{Bugs}, \emph{Author},
     \emph{version}, etc.

     \emph{Synopsis} must be the first section after the \texttt{Name}
     environment.

     Note: Do not use \LaTeX-macros in section names.
\item[\Bs subsection\{..\}] works as well as
\item[\Bs subsubsection\{..\}] those.
\item[\Bs emph\{..\}] \verb+\emph{example}+ is typeset as \emph{example}.
\item[\Bs textbf\{..\}] \verb+\textbf{example}+ is typeset as \textbf{example}.
\item[\Bs texttt\{..\}]\verb+\textt{example}+ is typeset as \texttt{example}.
\item[\Bs underline\{..\}]\verb+\underline{example}+ is typeset as
     \underline{example of underline}.
\item[\Bs date\{..\}] uses \verb+..+ as date.
\item[\Bs verb+..+] but only + is allowed as delimiter.
\item[\$$<$\$]  is typeset as $<$.
\item[\$$>$\$]  is typeset as $>$.
\item[\$$<=$\$] is typeset as $<=$.
\item[\$$>=$\$] is typeset as $>=$.
\item[\$$=$\$]  is typeset as $=$.
\item[\$$<>$\$] is typeset as $<>$.
\item[\$\Bs ge\$] is typeset as $\ge$.
\item[\$\Bs le\$] is typeset as $\le$.
\item[\$\Bs leftarrow\$] is typeset as $\leftarrow$.
\item[\$\Bs Leftarrow\$] is typeset as $\Leftarrow$.
\item[\$\Bs rightarrow\$] is typeset as $\rightarrow$.
\item[\$\Bs Rightarrow\$] is typeset as $\Rightarrow$.
\item[\Bs \{] is typeset as \{.
\item[\Bs \}] is typeset as \}.
\item[\Bs \$] is typeset as \$.
\item[\Bs \Dollar] is typeset as \Dollar, should be used inside macro
     arguments.
\item[\Bs \_] is typeset as \_.
\item[\Bs \&] is typeset as \&.
\item[\Bs \#] is typeset as \#.
\item[\Bs \%] is typeset as \%.
\item[\Bs ,] is typeset as smaller blank -\,- (between the two -)
\item[\Bs -]  is used to mark hyphenation in a word.
\item[\Bs \Bs] is typeset as a linebreak or marks the end of a column in the
     \texttt{Table} environment.
\item[\Bs ] (a \Bs\ followed by a blank) is typeset as a blank,
     although it cannot be used at the beginning of a line to make indentation
     (see the \texttt{\Bs SP} command).
\item[\Tilde] is typeset as a blank.
\item[\Bs copyright] is typeset as \copyright.
\item[\Bs noindent]
\item[\Bs hline] inside a \texttt{Table} environment.
\item[\Bs item] inside a \texttt{itemize}, \texttt{enumerate}, or
     \texttt{description} environment.
\item[\Bs today] \today\ (see also the \texttt{rcsinfo} \LaTeX\ package).
\item[\Bs ss,\Bs"a, ...] \Bs ss = \ss, \Bs"a = \"a, \Bs"o = \"o, \Bs"u = \"u,
     \Bs"A = \"A, \Bs"O = \"O, \Bs"U = \"U. It is allowed to surround these
     macros in \{ and \} in all places, even inside other macros, e.g.
     \begin{verbatim}
     \textbf{\"a\"o\"u\"A\"O\"U\ss}
     \textbf{\"a}{\"o}{\"u}{\"A}{\"O}{\"U}{\ss}}
     \textbf{�������}
     \end{verbatim}

     \textbf{\"a\"o\"u\"A\"O\"U\ss {\"a}{\"o}{\"u}{\"A}{\"O}{\"U}{\ss} �������}

     If these letters are used in their LATIN-1 8-bit coding, they are
     translated into the equivalent letter of the desired output format.
     E.g. \texttt{�} becomes \texttt{\&Auml}; in HTML and \texttt{@"A}
     in texinfo.
\end{description}

\subsection{Conditional Text}
%%%%%%%%%%%%%%%%%%%%%%%%%%%%%

\Prog{latex2man} preprocesses the \LaTeX\ input to allow text to be used
conditionally. A special sort of \LaTeX\ comment is used for that purpose.
\begin{itemize}\setlength{\itemsep}{0cm}
\item \Percent@\Percent\ IF \emph{condition} \Percent@\Percent
\item \Percent@\Percent\ ELSE \Percent@\Percent
\item \Percent@\Percent\ END-IF \Percent@\Percent
\end{itemize}

A line must contain only such a comment and nothing else.  \emph{condition} is
a boolean expression containing ``names'' and operators.  The names given with
the \OptArg{-C}{name} option have the value ``true'', while all other names
occuring in the expression are assumed to be ``false''.  If the evaluation of
the boolean expression results in the value ``true'', the text in the
``then''-part is used and the text in the optional ``else''-part is skipped
(and vice versa).  The \texttt{IF/ELSE/END-IF} may be nested.  As boolean
operators the following are allowed:

\begin{Table}{2}
\Bar\Bar & boolean or   \\
\&\&     & boolean and  \\
!        & negation
\end{Table}

( and ) for grouping are allowed.

\noindent
For example:\\
\Percent@\Percent\ IF abc                    \Percent@\Percent\\
\verb+    +abc set\\
\verb+    +\Percent@\Percent\ IF xyz         \Percent@\Percent\\
\verb+       +xyz set\\
\verb+    +\Percent@\Percent\ ELSE           \Percent@\Percent\\
\verb+       +xyz NOT set\\
\verb+    +\Percent@\Percent\ END-IF         \Percent@\Percent\\
\Percent@\Percent\ ELSE                      \Percent@\Percent\\
\verb+    +abc NOT set\\
\verb+    +\Percent@\Percent\ IF xyz \Bar\Bar\ !XYZ \Percent@\Percent\\
\verb+       +xyz OR !XYZ set\\
\verb+    +\Percent@\Percent\ ELSE           \Percent@\Percent\\
\verb+       +xyz OR !XYZ NOT set\\
\verb+    +\Percent@\Percent\ END-IF         \Percent@\Percent\\
\Percent@\Percent\ END-IF                    \Percent@\Percent

\noindent
Run this manual page through \Prog{latex2man} with e.g.\
\OptArg{-C}{'abc XYZ'} and have a look to the generated output.
(If simply running the \LaTeX-document\ through \LaTeX, all lines are shown in the
.dvi file).\\
%@% IF abc            %@%
   abc set\\
   %@% IF xyz         %@%
       xyz set\\
   %@% ELSE           %@%
       xyz NOT set\\
   %@% END-IF         %@%
%@% ELSE              %@%
    abc NOT set\\
    %@% IF xyz || !XYZ %@%
    xyz OR !XYZ set\\
    %@% ELSE          %@%
    xyz OR !XYZ NOT set
    %@% END-IF        %@%
%@% END-IF            %@%

\noindent
To check the conditional text feature, when \Prog{latex2man} is called with
\begin{description}
\item[\OptArg{-C}{HTML}]  the lines 1a, 2b, 3b, and 4b;
\item[\OptArg{-C}{TEXI}]  the lines 1b, 2a, 3b, and 4b;
\item[\OptArg{-C}{MAN}]   the lines 1b, 2b, 3a, and 4b;
\item[\OptArg{-C}{LATEX}] the lines 1b, 2b, 3b, and 4a;
\item[calling \LaTeX\ without preprocessing] all lines
\end{description}
should be shown:

%@% IF HTML %@%
1a. This text occurs only when viewing the HTML output.

%@% ELSE %@%
1b. The HTML conditional was not set.

%@% END-IF %@%

%@% IF TEXI %@%
2a. This text occurs only when viewing the TEXI output

%@% ELSE %@%
2b. The TEXI conditional was not set.

%@% END-IF %@%

%@% IF MAN %@%
3a. This text occurs only when viewing the MAN output

%@% ELSE %@%
3b. The MAN conditional was not set.

%@% END-IF %@%

%@% IF LATEX %@%
4a. This text occurs only when viewing the LATEX output

%@% ELSE %@%
4b. The LATEX conditional was not set.

%@% END-IF %@%

\subsection{Translation of User Defined Macros}
%%%%%%%%%%%%%%%%%%%%%%%%%%%%%%%%%%%%%%%%%%%%%%%

The user macro translation file (given by the \oOptArg{-t}{transfile}) contains
\Prog{Perl} commands specifying the translation of \LaTeX\ macros defined by
the user. These macros may have none, one or two arguments. The following code
is expected:

\begin{itemize}\setlength{\itemsep}{0cm}
\item Comments start with a \# up to the end of the line.
\item For a macro \verb+\foo+ with no arguments, the following code must be
     specified:
     \begin{description}\setlength{\itemsep}{0cm}
       \item[Translation to Man-Pages]\TEXbr
            \verb+$manMacro{'foo'}  = '...';+
       \item[Translation to HTML]\TEXbr
            \verb+$htmlMacro{'foo'} = '...';+
       \item[Translation to TexInfo]\TEXbr
            \verb+$texiMacro{'foo'} = '...';+
     \end{description}

     where \verb+...+ is the translation.

\item For a macro \verb+\foo{..}+ with one argument, the following code must be
     specified:
     \begin{description}\setlength{\itemsep}{0cm}
       \item[Translation to Man-Pages]\TEXbr
            \verb+$manMacro1a{'foo'}  = '...';+\\
            \verb+$manMacro1b{'foo'}  = '...';+
       \item[Translation to HTML]\TEXbr
            \verb+$htmlMacro1a{'foo'} = '...';+\\
            \verb+$htmlMacro1b{'foo'} = '...';+

       \item[Translation to TexInfo]\TEXbr
            \verb+$texiMacro1a{'foo'} = '...';+\\
            \verb+$texiMacro1b{'foo'} = '...';+
     \end{description}

     where \verb+...+ is the translation. The \verb+1a+ code is used before the
     argument, while \verb+1b+ is typeset after the argument is set.

\item For a macro \verb+\foo{..}{..}+ with two arguments, the following code
     must be specified:
     \begin{description}\setlength{\itemsep}{0cm}
     \item[Translation to Man-Pages]\TEXbr
          \verb+$manMacro2a{'foo'}  = '...';+\\
          \verb+$manMacro2b{'foo'}  = '...';+\\
          \verb+$manMacro2c{'foo'}  = '...';+
     \item[Translation to HTML]\TEXbr
          \verb+$htmlMacro2a{'foo'} = '...';+\\
          \verb+$htmlMacro2b{'foo'} = '...';+\\
          \verb+$htmlMacro2c{'foo'} = '...';+
     \item[Translation to TexInfo]\TEXbr
          \verb+$texiMacro2a{'foo'} = '...';+\\
          \verb+$texiMacro2b{'foo'} = '...';+\\
          \verb+$texiMacro2c{'foo'} = '...';+
     \end{description}

     where \verb+...+ is the translation. The \verb+2a+ code is used before the
     first argument, \verb+2b+ between the two arguments and \verb+2c+ is
     typeset after the second argument is set.
\item The file \File{latex2man.trans} contains some example code.
\end{itemize}

\subsection{Verbatim Environment}
%%%%%%%%%%%%%%%%%%%%%%%%%%%%%%%%%

\begin{verbatim}
This
    {is}
        \texttt{a}
                  $test$
                         _of_
verbatim
<this is no HTML tag> and no @* TexInfo command
\end{verbatim}

\subsection{Subsection works}
%%%%%%%%%%%%%%%%%%%%%%%%%%%%%

This is a \verb+\subsection+.

\subsubsection{Subsubsection works}
%%%%%%%%%%%%%%%%%%%%%%%%%%%%%%%%%%%

This is a \verb+\subsubsection+.

\subsubsection{Subsubsection still works}
%%%%%%%%%%%%%%%%%%%%%%%%%%%%%%%%%%%%%%%%%

This is another \verb+\subsubsection+.


\subsection{General Remarks}
%%%%%%%%%%%%%%%%%%%%%%%%%%%%

\begin{enumerate}\setlength{\itemsep}{0cm}
\item Empty lines are typeset as paragraph separators.
\item The arguments of the \LaTeX\ commands must not be split over several
     lines.
\item Do not nest calls to macros.
\item Except the mentioned environment and macros, the usage of other \LaTeX\
     environments or macros are not translated. Their usage will cause garbage
     in the output.
\item \Prog{latex2man} requires Perl version $>=$ 5.0004\_03.
\item If you want to install the system with the distributed \File{Makefile},
     you need GNU-\Prog{make}. If you don't have it, you should execute the
     steps shown in the \File{Makefile} manually.
\end{enumerate}

\subsection{Some Bug Fix Tests}
%%%%%%%%%%%%%%%%%%%%%%%%%%%%%%%

% TODO: HTML: a blank line after \begin{description} should not emit a <p>
\begin{description}
\item[Leading . and ']
%%%%%%%%%%%%%%%%%%%%%%
Now leading . and ' in generation troff output should work propperly,
since a \verb+\&+ is added. Therfore the \verb+\Dot+ macro has been deleted.\\
Thanks to \Email{Frank.Schilder@Mathematik.Tu-Ilmenau.De}.\\
Testcase 1:
\begin{description}
\item['\Bs n'] ...
\end{description}
Testcase 2:\\
\File{.foobar}
Testcase 3:\\
\Dots

abc \Dots abc . efg ' 123

\item[\% in verbatim] A \% in a \Bs verb and verbatim-environment was not
     emitted correctly.  Thanks to Aleksey Nogin \Email{nogin@cs.caltech.edu}
     for the bug report and bug fix.

\verb+ % abc+

\begin{verbatim}
   % abc %
\end{verbatim}

but ignore comments following this: % this should be ignored

\end{description}

\section{Requirements}
%%%%%%%%%%%%%%%%%%%%%%

\begin{description}\setlength{\itemsep}{0cm}
\item[Perl] \Prog{latex2man} requires Perl version $>=$ 5.0004\_03.
\item[Make] If you want to install the system with the distributed
     \File{Makefile}, you need GNU-\Prog{make}. If you don't have it, you
     should execute the steps shown in the \File{Makefile} manually.
\item[\LaTeX] \LaTeX 2e is required.
\end{description}

\section{Changes}
%@% IF LATEX %@%
{\small\verbatiminput{CHANGES}}
%@% ELSE %@%
Please check the file \URL{latex2man-CHANGES} for the list of changes.
%@% END-IF %@%

\section{Version}
%%%%%%%%%%%%%%%%%

Version: \Version\ of \Date.

\section{License and Copyright}
%%%%%%%%%%%%%%%%%%%%%%%%%%%%%%%

\begin{description}
\item[Copyright] \copyright\ 1998, Dr.~J�rgen Vollmer, Viktoriastra�e 15,
     D-76133 Karlsruhe, Germany, \Email{Juergen.Vollmer@informatik-vollmer.de}


     The actual version of  \Prog{Latex2man} may be found on my homepage\\
     \URL{http://www.informatik-vollmer.de/software/latex2man.html}.

\item[License] This program can be redistributed and/or modified under the
     terms of the LaTeX Project Public License Distributed from CTAN archives
     in directory \File{macros/latex/base/lppl.txt}; either version 1 of the
     License, or any later version.

\item[Misc] If you find this software useful, please send me a postcard from
     the place where you are living.
\end{description}

\section{Author}
%%%%%%%%%%%%%%%%

\noindent
Dr.~J�rgen Vollmer                      \\
Viktoriastra�e 15                       \\
D-76133 Karlsruhe                       \\
Email: \Email{Juergen.Vollmer@informatik-vollmer.de}  \\
WWW: \URL{http://www.informatik-vollmer.de}.

\LatexManEnd

\end{document}
