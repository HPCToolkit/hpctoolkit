% -*- Mode: latex; -*-
% $HeadURL$
% $Id$

This chapter describes the mechanics of using \hpcrun{} and \hpclink{}
to profile an application and collect performance data.  For advice on
choosing sampling events, scaling studies, etc, see
Chapter~\ref{chpt:effective-performance-analysis} on {\it Effective
Strategies for Analyzing Program Performance}.

\section{Using \hpcrun{}}

The \hpcrun{} launch script is used to run an application and collect
performance data for {\it dynamically linked\/} binaries.  For
dynamically linked programs, this requires no change to the program
source and no change to the build procedure.  You should build your
application natively at full optimization.  \hpcrun{} inserts its
profiling code into the application at runtime via \verb|LD_PRELOAD|.

The basic options for \hpcrun{} are {\tt -e} (or {\tt --event}) to
specify a sampling source and rate and {\tt -t} (or {\tt --trace}) to
turn on tracing.  Sample sources are specified as `{\tt event@period}'
where {\tt event} is the name of the source and {\tt period} is the
period (threshold) for that event, and this option may be used
multiple times.  Note that a higher period implies a lower rate of
sampling.  The basic syntax for profiling an application with
\hpcrun{} is:

\begin{quote}
\begin{verbatim}
hpcrun -t -e event@period ... app arg ...
\end{verbatim}
\end{quote}

For example, to profile an application and sample every 15,000,000
total cycles and every 400,000 L2 cache misses you would use:

\begin{quote}
\begin{verbatim}
hpcrun -e PAPI_TOT_CYC@15000000 -e PAPI_L2_TCM@400000 app arg ...
\end{verbatim}
\end{quote}

The units for the {\tt WALLCLOCK} sample source are in microseconds,
so to sample an application with tracing every 5,000 microseconds
(200~times/second), you would use:

\begin{quote}
\begin{verbatim}
hpcrun -t -e WALLCLOCK@5000 app arg ...
\end{verbatim}
\end{quote}

\hpcrun{} stores its raw performance data in a {\it measurements}
directory with the program name in the directory name.  On systems
with a batch job scheduler (eg, PBS) the name of the job is appended
to the directory name.

\begin{quote}
\begin{verbatim}
hpctoolkit-app-measurements[-jobid]
\end{verbatim}
\end{quote}

It is best to use a different measurements directory for each run.
So, if you're using \hpcrun{} on a local workstation without a job
launcher, you can use the `{\tt -o dirname}' option to specify an
alternate directory name.

For programs that use their own launch script (eg, {\tt mpirun} or
{\tt mpiexec} for MPI), put the application's run script on the
outside (first) and \hpcrun{} on the inside (second) on the command
line.  For example,

\begin{quote}
\begin{verbatim}
mpirun -n 4 hpcrun -e PAPI_TOT_CYC@15000000 mpiapp arg ...
\end{verbatim}
\end{quote}

Note that \hpcrun{} is intended for profiling dynamically linked {\it
binaries}.  It will not work well if used to launch a shell script.
At best, you would be profiling the shell interpreter, not the script
commands, and sometimes this will fail outright.

It is possible to use \hpcrun{} to launch a statically linked binary,
but there are two problems with this.  First, it is still necessary to
build the binary with \hpclink{}.  Second, static binaries are
commonly used on parallel clusters that require running the binary
directly and do not accept a launch script.  However, if your system
allows it, and if the binary was produced with \hpclink, then
\hpcrun{} will set the correct environment variables for profiling
statically or dynamically linked binaries.  All that \hpcrun{} really
does is set some environment variables (including \verb|LD_PRELOAD|)
and {\tt exec} the binary.

% ===========================================================================

\section{Using \hpclink{}}

For now, see Chapter~\ref{chpt:statically-linked-apps} on {\it
Monitoring Statically Linked Applications}.

% ===========================================================================

\section{Sample Sources}

This section covers the details of some individual sample sources.  To
see a list of the available sample sources and events that \hpcrun{}
supports, use `{\tt hpcrun -L}' (dynamic) or set
`\verb|HPCRUN_EVENT_LIST=LIST|' (static).  Note that on systems with
separate compute nodes, you will likely need to run this on one of the
compute nodes.

\subsection{PAPI}

PAPI, the Performance API, is a library for providing access to the
hardware performance counters.  This is an attempt to provide a
consistent, high-level interface to the low-level performance
counters.  PAPI is available from the University of Tennessee at:

\begin{quote}
\url{http://icl.cs.utk.edu/papi/}
\end{quote}

PAPI focuses mostly on in-core CPU events: cycles, cache misses,
floating point operations, mispredicted branches, etc.  For example,
the following command samples total cycles and L2 cache misses.

\begin{quote}
\begin{verbatim}
hpcrun -e PAPI_TOT_CYC@15000000 -e PAPI_L2_TCM@400000 app arg ...
\end{verbatim}
\end{quote}

The precise set of PAPI preset and native events is highly system
dependent.  Commonly, there are events for machine cycles, cache
misses, floating point operations and other more system specific
events.  However, there are restrictions both on how many events can
be sampled at one time and on what events may be sampled together and
both restrictions are system dependent.  Table~\ref{tab:papi-events}
contains a list of commonly available PAPI events.

To see what PAPI events are available on your system, use the
\verb|papi_avail| command from the {\tt bin} directory in your PAPI
installation.  The event must be both available and not derived to be
usable for sampling.  The command \verb|papi_native_avail| displays
the machine's native events.  Note that on systems with separate
compute nodes, you normally need to run \verb|papi_avail| on one of
the compute nodes.

\begin{table}
\begin{center}
\begin{tabular}{|l|l|}
\hline
\verb|PAPI_BR_INS| & Branch instructions \\
\hline
\verb|PAPI_BR_MSP| & Conditional branch instructions mispredicted \\
\hline
\verb|PAPI_FP_INS| & Floating point instructions \\
\hline
\verb|PAPI_L1_DCA| & Level 1 data cache accesses \\
\hline
\verb|PAPI_L1_DCM| & Level 1 data cache misses \\
\hline
\verb|PAPI_L1_ICH| & Level 1 instruction cache hits \\
\hline
\verb|PAPI_L1_ICM| & Level 1 instruction cache misses \\
\hline
\verb|PAPI_L2_DCA| & Level 2 data cache accesses \\
\hline
\verb|PAPI_L2_ICM| & Level 2 instruction cache misses \\
\hline
\verb|PAPI_L2_TCM| & Level 2 cache misses \\
\hline
\verb|PAPI_LD_INS| & Load instructions \\
\hline
\verb|PAPI_SR_INS| & Store instructions \\
\hline
\verb|PAPI_TLB_DM| & Data translation lookaside buffer misses \\
\hline
\verb|PAPI_TOT_CYC| & Total cycles \\
\hline
\verb|PAPI_TOT_IIS| & Instructions issued \\
\hline
\verb|PAPI_TOT_INS| & Instructions completed \\
\hline
\end{tabular}
\end{center}
\caption{Some commonly available PAPI events.
The exact set of available events is system dependent.}
\label{tab:papi-events}
\end{table}

When selecting the period for PAPI events, aim for a rate of
approximately a few hundred samples per second.  So, roughly several
million or tens of million for total cycles or a few hundred thousand
for cache misses.  PAPI and \hpcrun{} will tolerate sampling rates as
high as 1,000 or even 10,000 samples per second (or more).  But rates
higher than a few thousand samples per second will only drive up the
overhead and distort the running of your program.  It won't give more
accurate results.

Earlier versions of PAPI required a separate patch (perfmon or
perfctr) for the Linux kernel.  But beginning with kernel 2.6.32,
support for accessing the performance counters (perf events) is now
built in to the standard Linux kernel.  This means that on kernels
2.6.32 or later, PAPI can be compiled and run entirely in user land
without patching the kernel.  PAPI is highly recommended and well
worth building if it is not already installed on your system.

\subsection{Wallclock}

The {\tt WALLCLOCK} sample source is based on the \verb|ITIMER_PROF|
interval timer.  Normally, \verb|PAPI_TOT_CYC| is just as good as {\tt
WALLCLOCK} and often better, but {\tt WALLCLOCK} can be used on
systems where PAPI is not available.  The units are in microseconds,
so the following example will sample {\tt app} approximately 200
times per second.

\begin{quote}
\begin{verbatim}
hpcrun -e WALLCLOCK@5000 app arg ...
\end{verbatim}
\end{quote}

Note that the maximum interrupt rate from itimer is limited by the
system's Hz rate, commonly 1,000 cycles per second, but may be lower.
That is, {\tt WALLCLOCK@10} will not generate any higher sampling rate
than {\tt WALLCLOCK@1000}.  However, on IBM BlueGene, itimer is not
bound by the Hz rate and so sampling rates faster than 1,000 per
second are possible.

Do not use both PAPI and {\tt WALLCLOCK} events at the same time.
That would cause PAPI samples to be interrupted by itimer signals and
vice versa, and \hpcrun{} is not equipped to handle this case.

\subsection{Memleak}

The {\tt MEMLEAK} sample source records the number of bytes allocated
and freed by the program along with their dynamic context.  {\tt
MEMLEAK} is a synchronous sample source.  That is, it does not
generate asynchronous interrupts.  Instead, it overrides the malloc
family of functions ({\tt malloc}, {\tt calloc}, {\tt realloc} and
{\tt free} plus {\tt memalign}, {\tt posix\_memalign} and {\tt
valloc}).

{\tt MEMLEAK} allows you to find locations in your program that
allocate memory that is never freed.  But note that failure to free a
memory location does not necessarily imply that location has leaked
(missing a pointer to the memory).  It is common for programs to
allocate memory that is used throughout the lifetime of the process
and not explicitly free it.

To include this source, use the {\tt MEMLEAK} event (no period).
{\tt MEMLEAK} uses a separate library, so in the static case, use the
{\tt --memleak} option to add this library.  For example,

\begin{quote}
\begin{tabular}{@{}cl}
(dynamic) & \verb|hpcrun -e MEMLEAK app arg ...| \\
(static)  & \verb|hpclink --memleak gcc -g -O -static -o app file.c ...| \\
& \verb|export HPCRUN_EVENT_LIST=MEMLEAK| \\
& \verb|app arg ...|
\end{tabular}
\end{quote}

If a program allocates and frees many small regions, the {\tt MEMLEAK}
source may result in a high overhead.  In this case, you may reduce
the overhead by using the memleak probability option to record only a
fraction of the mallocs.  For example, to monitor 10\% of the mallocs,
use:

\begin{quote}
\begin{tabular}{@{}cl}
(dynamic) & \verb|hpcrun -e MEMLEAK --memleak-prob 0.10 app arg ...| \\
(static)  & \verb|export HPCRUN_EVENT_LIST=MEMLEAK| \\
& \verb|export HPCRUN_MEMLEAK_PROB=0.10| \\
& \verb|app arg ...|
\end{tabular}
\end{quote}

It might appear that if you monitor only 10\% of the program's
mallocs, then you would have only a 10\% chance of finding the leak.
But if a program leaks memory, then it's likely that it does so many
times, all from the same source location.  And you only have to find
that location once.  So, this option can be a useful tool if the
overhead of recording all mallocs is prohibitive.

% ===========================================================================

\section{Process Fraction}

Although \hpcrun{} can profile parallel jobs with thousands or tens of
thousands of processes, there are two scaling problems that become
prohibitive beyond a few thousand cores.  First, \hpcrun{} writes the
measurement data for all of the processes into a single directory.
This results in one file per process plus one file per thread (two
files per thread if using tracing).  Unix file systems are not
equipped to handle directories with many tens or hundreds of thousands
of files.  Second, the sheer volume of data can overwhelm the viewer
when the size of the database far exceeds the amount of memory on the
machine.

The solution is to sample only a fraction of the processes.  That is,
you can run an application on many thousands of cores but record data
for only a few hundred processes.  The other processes run the
application but do not record any measurement data.  This is what the
process fraction option ({\tt -f} or {\tt --process-fraction}) does.
For example, to monitor 10\% of the processes, use:

\begin{quote}
\begin{tabular}{@{}cl}
(dynamic) & \verb|hpcrun -f 0.10 -e event@period app arg ...| \\
(dynamic) & \verb|hpcrun -f 1/10 -e event@period app arg ...| \\
(static)  & \verb|export HPCRUN_EVENT_LIST='event@period'| \\
& \verb|export HPCRUN_PROCESS_FRACTION=0.10| \\
& \verb|app arg ...|
\end{tabular}
\end{quote}

With this option, each process generates a random number and records
its measurement data with the given probability.  The process fraction
(probability) may be written as a decimal number (0.10) or as a
fraction (1/10) between 0 and 1.  So, in the above example, all three
cases would record data for approximately 10\% of the processes.  Aim
for a number of processes in the hundreds.

% ===========================================================================

\section{Starting and Stopping Sampling}

\HPCToolkit{} supports an API for the application to start and stop
sampling.  This is useful if the application wants to profile only a
subset of the program and ignore the rest.  This interface is process
wide, not thread specific.  And technically, when sampling is turned
off, \hpcrun{} continues to receive interrupts, but it doesn't unwind
the call stack or record samples.  This API supports the following
functions.

\begin{quote}
\begin{verbatim}
void hpctoolkit_sampling_start(void);
void hpctoolkit_sampling_stop(void);
\end{verbatim}
\end{quote}

To use this interface, put the above function calls into your program
where you want sampling to start and stop.  Remember, starting and
stopping apply process wide.  For C/C++, include the following header
file from the \HPCToolkit{} {\tt include} directory.

\begin{quote}
\begin{verbatim}
#include <hpctoolkit.h>
\end{verbatim}
\end{quote}

Compile your application with {\tt libhpctoolkit} with {\tt -I} and
{\tt -L} options for the include and library paths.  For example,

\begin{quote}
\begin{verbatim}
gcc -I /path/to/hpctoolkit/include app.c ... \
    -L /path/to/hpctoolkit/lib/hpctoolkit -lhpctoolkit ...
\end{verbatim}
\end{quote}

The {\tt libhpctoolkit} library provides weak symbol no-op definitions
for the start and stop functions.  For dynamically linked programs, be
sure to include {\tt -lhpctoolkit} on the link line (otherwise your
program won't link).  For statically linked programs, \hpclink{} adds
strong symbol definitions for these functions.  So, {\tt -lhpctoolkit}
is not necessary in the static case, but it doesn't hurt.

To run the program, set the \verb|LD_LIBRARY_PATH| environment
variable to include the \HPCToolkit{} {\tt lib/hpctoolkit} directory.
This step is only needed for dynamically linked programs.

\begin{quote}
\begin{verbatim}
export LD_LIBRARY_PATH=/path/to/hpctoolkit/lib/hpctoolkit
\end{verbatim}
\end{quote}

Note that sampling is initially turned on until the program turns it
off.  If you want it initially turned off, then use the {\tt -ds} (or
{\tt --delay-sampling}) option for \hpcrun{} (dynamic) or set the
\verb|HPCRUN_DELAY_SAMPLING| environment variable (static).

\begin{quote}
\begin{tabular}{@{}cl}
(dynamic) & \verb|hpcrun -ds -e event@period app arg ...|  \\
(static)  & \verb|export HPCRUN_EVENT_LIST='event@period'| \\
& \verb|export HPCRUN_DELAY_SAMPLING=1| \\
& \verb|app arg ...|
\end{tabular}
\end{quote}

