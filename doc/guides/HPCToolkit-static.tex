%
% $Id$
%
\documentclass[12pt]{article}

\usepackage{fancyvrb}
\usepackage{fullpage}
\usepackage{moreverb}
\usepackage{hyperref}
\usepackage{verbatim}

\setlength{\topmargin}{0in}
\setlength{\oddsidemargin}{0in}
\setlength{\evensidemargin}{0in}
\setlength{\textwidth}{6.5in}
% \setlength{\parskip}{4pt}
% \setlength{\parindent}{0pt}

\newcommand{\HPCToolkit}{\textsc{HPCToolkit}}
\newcommand{\hpcrun}{\texttt{hpcrun}}
\newcommand{\hpclink}{\texttt{hpclink}}
\newcommand{\hpcstruct}{\texttt{hpcstruct}}
\newcommand{\hpcprof}{\texttt{hpcprof}}
\newcommand{\hpcviewer}{\texttt{hpcviewer}}

\newcommand{\question}[1]{\vspace{4pt}\par\noindent{\bf Q: #1}}
\newcommand{\answer}{\par\vspace{2pt}\noindent{\bf A:}}

\begin{document}

\title{Using {\sc HPCToolkit} with Statically-Linked Programs}
\author{The \HPCToolkit\ Team}
\maketitle

\begin{abstract}
\HPCToolkit{} is an integrated suite of tools that supports
measurement, analysis, attribution and presentation of application
performance for sequential and parallel programs.  This paper
describes how to link and run \HPCToolkit\ with statically-linked
applications.
\end{abstract}

\section{Introduction}

Some large-scale parallel systems, for example Compute-Node Linux and
BG/P, do not support dynamically-linked executables on the compute
nodes and run only fully statically-linked binaries.  On these
systems, you don't use \hpcrun\ to run your program.  Instead, you use
\hpclink\ to build a version of your program with the \HPCToolkit\
profiler linked in.

\section{Linking with \hpclink }

In the static case, \hpclink\ is used to link the \HPCToolkit\
libraries into your application.  This does not require any
source-code modifications, but it does involve a small change to the
build procedure, namely the final linking step.  You continue to make
all the object ({\tt .o}) files exactly as before, but in the last
step, you use \hpclink\ to link in the \HPCToolkit\ code.

First, find out how to build your binary natively, without
\HPCToolkit, and look for the last step in the build process, the
command that produces the single, statically-linked binary.  Then, run
the same command line, except with \hpclink\ in front of it.

For example, suppose your binary is {\tt myprog} and the last step in
your {\tt Makefile} combines various object files and libraries as
follows.
%
\begin{quote}
\verb|mpicc -o myprog -static file.o ... -l<lib> ...|
\end{quote}
%
Then, you would build a version with \HPCToolkit\ linked in with the
following command line.
%
\begin{quote}
\verb|hpclink mpicc -o myprog -static file.o ... -l<lib> ...|
\end{quote}

In practice, you may want to edit your {\tt Makefile} to always build
two versions of your program, perhaps naming them {\tt myprog} and
{\tt myprog.hpc}.

\section{Running the Statically-Linked Binary}

In the dynamic case, the \hpcrun\ script sets environment variables to
pass the script options to the \HPCToolkit\ library, but \hpcrun\ is
not used in the static case.  Instead, you will need to set these
variables yourself in your launch script.

About the only variable you will need to set is
{\tt CSPROF\_OPT\_EVENT}, which controls the sampling events.  This
should be set to a space-separated list of {\tt EVENT@COUNT} pairs.
For example, in a PBS script in Bourne shell or bash syntax, you might
write:

\begin{quote}
\begin{verbatim}
#!/bin/sh
#PBS -l size=64
#PBS -l walltime=01:00:00
cd $PBS_O_WORKDIR
export CSPROF_OPT_EVENT="PAPI_TOT_CYC@4000000 PAPI_L2_TCM@400000"
aprun -n 64 ./myprog arg ...
\end{verbatim}
\end{quote}
% $ Artificially end math mode.

On the BG/P system, you use the {\tt --env} option to pass environment
variables.  For example, you might submit a job with:

\begin{quote}
\begin{verbatim}
qsub -t 60 -n 64 --env CSPROF_OPT_EVENT="PAPI_TOT_CYC@4000000" \
    /path/to/myprog arg ...
\end{verbatim}
\end{quote}

\section{Troubleshooting}

With some compilers you may need to disable the interprocedural
optimization.  Statically, \hpclink\ gets its hooks into a program
with the {\tt ld} option {\tt --wrap} (see the ld(1) man page).  The
interprocedural optimization interferes with the {\tt --wrap} option
and prevents \hpclink\ from linking the \HPCToolkit\ libraries.

\end{document}
