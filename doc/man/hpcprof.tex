%% $Id$

%%%%%%%%%%%%%%%%%%%%%%%%%%%%%%%%%%%%%%%%%%%%%%%%%%%%%%%%%%%%%%%%%%%%%%%%%%%%%
%%%%%%%%%%%%%%%%%%%%%%%%%%%%%%%%%%%%%%%%%%%%%%%%%%%%%%%%%%%%%%%%%%%%%%%%%%%%%

\documentclass[english]{article}
\usepackage[latin1]{inputenc}
\usepackage{babel}
\usepackage{verbatim}

%% do we have the `hyperref package?
\IfFileExists{hyperref.sty}{
   \usepackage[bookmarksopen,bookmarksnumbered]{hyperref}
}{}

%% do we have the `fancyhdr' or `fancyheadings' package?
\IfFileExists{fancyhdr.sty}{
\usepackage[fancyhdr]{latex2man}
}{
\IfFileExists{fancyheadings.sty}{
\usepackage[fancy]{latex2man}
}{
\usepackage[nofancy]{latex2man}
\message{no fancyhdr or fancyheadings package present, discard it}
}}

%% do we have the `rcsinfo' package?
\IfFileExists{rcsinfo.sty}{
\usepackage[nofancy]{rcsinfo}
\rcsInfo $Id$
\setDate{\rcsInfoLongDate}
}{
\setDate{2021/09/11}
\message{package rcsinfo not present, discard it}
}

\setVersionWord{Version:}  %%% that's the default, no need to set it.
\setVersion{@PACKAGE_VERSION@}

%%%%%%%%%%%%%%%%%%%%%%%%%%%%%%%%%%%%%%%%%%%%%%%%%%%%%%%%%%%%%%%%%%%%%%%%%%%%%
%%%%%%%%%%%%%%%%%%%%%%%%%%%%%%%%%%%%%%%%%%%%%%%%%%%%%%%%%%%%%%%%%%%%%%%%%%%%%

\begin{document}

\begin{Name}{1}{hpcprof}{The HPCToolkit Performance Tools}{The HPCToolkit Performance Tools}{hpcprof:\\ Analysis and Attribution of Call Path Performance Measurements}

\Prog{hpcprof} analyzes call path profile performance measurements
and attributes them to static source code structure.
See \HTMLhref{hpctoolkit.html}{\Cmd{hpctoolkit}{1}} for an overview of \textbf{HPCToolkit}.

\end{Name}

%%%%%%%%%%%%%%%%%%%%%%%%%%%%%%%%%%%%%%%%%%%%%%%%%%%%%%%%%%%%%%%%%%
\section{Synopsis}

\Prog{hpcprof} \oOpt{options} \Arg{measurements}...

\Prog{<mpi-launcher>}... \Prog{hpcprof-mpi} \oOpt{options} \Arg{measurements}...

%%%%%%%%%%%%%%%%%%%%%%%%%%%%%%%%%%%%%%%%%%%%%%%%%%%%%%%%%%%%%%%%%%
\section{Description}

\Prog{hpcprof} analyzes call path profile performance measurements,
attributes them to static source code structure,
and generates an experiment database for use with \HTMLhref{hpcviewer.html}{\Cmd{hpcviewer}{1}}.

\Prog{hpcprof-mpi} additionally parallelizes the analysis across multiple compute
nodes, using MPI for communication.
This is usually not needed except for especially large-scale executions (more than about 100,000 threads).
Using 8-10 compute nodes is recommended as larger numbers tend to not improve the analysis speed.

\Prog{hpcprof}/\Prog{hpcprof-mpi} expects a list of \emph{measurements},
each of which is either a call path profile directory or an individual profile file.
\Prog{hpcprof}/\Prog{hpcprof-mpi} will find only source files whose paths are recorded in an application binary, shared library, or GPU binary
as an absolute path still present in the file system or a relative path with respect to the current working directory.
If \HTMLhref{hpcstruct.html}{\Cmd{hpcstruct}{1}} was run on the measurements directory, no \textbf{-S} arguments are needed.


%%%%%%%%%%%%%%%%%%%%%%%%%%%%%%%%%%%%%%%%%%%%%%%%%%%%%%%%%%%%%%%%%%
\section{Arguments}

\begin{Description}
\item[\Arg{measurements}...] A sequence of file system paths,
each specifying a call path profile directory or an individual profile file.
\end{Description}

Default values for an option's optional arguments are shown in \{\}.

\subsection{Options: Informational}

\begin{Description}

\item[\Opt{-v}, \Opt{--verbose}]
Print additional messages to stderr.

\item[\Opt{-V}, \Opt{--version}]
Print version information.

\item[\Opt{-h}, \Opt{--help}]
Print help.

\item[\OptArg{-j}{threads}]
Perform analysis with \Arg{threads} threads.  \{<all available>\}

\end{Description}

\subsection{Options: Source Code and Static Structure}

\begin{Description}
\item[\OptArg{--name}{name}, \OptArg{--title}{name}]
Set the database's name (title) to \Arg{name}.

\item[\OptArg{-S}{file}, \OptArg{--structure}{file}]
Use the structure file \Arg{file} produced by \HTMLhref{hpcstruct.html}{\Cmd{hpcstruct}{1}}
to identify source code elements for attribution of performance.
This option may be given multiple times,
e.g. to provide structure for shared libraries in addition to the application executable.
If \HTMLhref{hpcstruct.html}{\Cmd{hpcstruct}{1}} was run on the measurements directory, no such arguments are needed.

\item[\OptArg{-R}{'old-path=new-path'}, \OptArg{--replace-path}{'old-path=new-path'}]
Replace every instance of \Arg{old-path} by \Arg{new-path}
in all paths for which \Arg{old-path} is a prefix of a path to a binary measured by HPCToolkit
or a source file used to produce a binary measured by HPCToolkit.
Use '\Bs'\ to escape instances of '=' within a path.
This option may be given multiple times.
\medskip
Use this option when a profile or binary contains references to files no
longer present at their original path.
For instance, a library may have been compiled by a system administrator and
distributed with line map information containing file paths that point to a
build directory that no longer exists.
If you can locate a copy of the source code for the library, you can unpack
it anywhere and provide a -R option that maps the prefix of the build
directory to the prefix of the directory where you unpacked a copy of the library sources.

\item[\OptArg{--only-exe}{filename}]
Only include measurements for executables with the given filename. This option may be
given multiple times to indicate multiple executables to include.
\medskip
Use this option when the application is measured through a wrapper script
(e.g. \texttt{hpcrun script.sh}) and you don't want to include the shell used to run the script in the
resulting performance database.

\end{Description}

\subsection{Options: Metrics}

\begin{Description}

\item[\OptArg{-M}{metric}, \OptArg{--metric}{metric}]
Compute the specified metrics, where \Arg{metric} is one of the following:
  \begin{itemize}
  \item[sum] Sum over threads/processes
  \item[stats] Sum, Mean, StdDev (standard deviation), CoefVar (coefficient of variation),
  Min, Max over threads/processes
  \item[thread] per-thread/process metrics
  \end{itemize}

The default metric is \Prog{sum}.
This option may be given multiple times.

\end{Description}

\subsection{Options: Output}

\begin{Description}

\item[\OptArg{-o}{db-path}, \OptArg{--db}{db-path}, \OptArg{--output}{db-path}]
Write the computed experiment database to \Arg{db-path}.
The default path is \File{./hpctoolkit-$<$application$>$-database}.

\end{Description}


%%%%%%%%%%%%%%%%%%%%%%%%%%%%%%%%%%%%%%%%%%%%%%%%%%%%%%%%%%%%%%%%%%
\section{Examples}

%\begin{enumerate}
%\item


%\end{enumerate}

%%%%%%%%%%%%%%%%%%%%%%%%%%%%%%%%%%%%%%%%%%%%%%%%%%%%%%%%%%%%%%%%%%
%\section{Notes}


%%%%%%%%%%%%%%%%%%%%%%%%%%%%%%%%%%%%%%%%%%%%%%%%%%%%%%%%%%%%%%%%%%
\section{See Also}

\HTMLhref{hpctoolkit.html}{\Cmd{hpctoolkit}{1}}.

%%%%%%%%%%%%%%%%%%%%%%%%%%%%%%%%%%%%%%%%%%%%%%%%%%%%%%%%%%%%%%%%%%
\section{Version}

Version: \Version

%%%%%%%%%%%%%%%%%%%%%%%%%%%%%%%%%%%%%%%%%%%%%%%%%%%%%%%%%%%%%%%%%%
\section{License and Copyright}

\begin{description}
\item[Copyright] \copyright\ 2002-2023, Rice University.
\item[License] See \File{LICENSE}.
\end{description}

%%%%%%%%%%%%%%%%%%%%%%%%%%%%%%%%%%%%%%%%%%%%%%%%%%%%%%%%%%%%%%%%%%
\section{Authors}

\noindent
Rice University's HPCToolkit Research Group \\
Email: \Email{hpctoolkit-forum =at= rice.edu} \\
WWW: \URL{http://hpctoolkit.org}.

\LatexManEnd

\end{document}

%% Local Variables:
%% eval: (add-hook 'write-file-hooks 'time-stamp)
%% time-stamp-start: "setDate{ "
%% time-stamp-format: "%:y/%02m/%02d"
%% time-stamp-end: "}\n"
%% time-stamp-line-limit: 50
%% End:
